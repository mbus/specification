\svnInfo $Id$

\section{Specifications}
\label{sec:spec}

\subsection{Granularity}
\label{sec:spec-granularity}
\bus sends data in units of 8~bit bytes. Transmissions must be modulo~8 bits
in length, that is they must end on byte boundaries (for rationale,
see~\nameref{sec:design-granularity}).

\subsection{Endianness: Byte-Level}
\bus sends from Byte 0{\ldots}Byte N.

\subsection{Endianness: Bit-Level}
\bus sends bytes most significant bit first (bit 7{\ldots}0).

\subsection{Minimum Message Length}
The minimum legal message on \bus is {\em zero}~bits long. A zero length
message can be used to query whether a device is present on the bus. A node
receiving a zero length message addressed to it {\bf must} ACK the message.
Whether higher layers are indicated of a query message is implementation
dependent.

See~\ref{sec:spec-interrupt}~\nameref{sec:spec-interrupt} for details on the
minimum uninterruptable message length.

\subsection{Minimum Maximum Message Length}
All \bus control nodes must permit a minimum of 1~K (1024~bits) of data to be
transmitted before aborting a transaction due to a hung transmitter.

The message length counter shall count the number of positive clock edges
received after Begin~Transmission at the control node's {\tt CLK\_IN} port up
to and including the last bit latched (Request~Interrupt in
Figure~\ref{fig:interrupt}). The counter shall be an {\em inclusive} counter,
that is the control node shall not Interrupt until it receives the $N+1$th
bit.

Commands to query and configure the upper bound are specified in
\nameref{sec:channel-2}.

\subsection{Retransmission}
There is no hardware re-transmission primitive. The hardware provides a
transaction-level message acknowledgement, indicating whether the complete
message was received or not. Retransmission of failed messages is left to
software.

\subsection{Minimum Buffer Size}
All \bus nodes are required to support both short and full
addresses~(\ref{sec:spec-address}~\nameref{sec:spec-address}). All \bus nodes
are further required to accept a minimum of 32~bits (4~bytes) of data in an
individual transmission.

\subsection{Buffer Overflow / Flow Control}
On the average, the expectation in a \bus system is that a transmitting node
knows the capacity of the receiving node. If a receiving node does not have
enough space for the current transmission, it {\bf must} Interrupt the current
transmission and indicate a receiver error (Control~Bits:~{\tt 01}).

The receiver must Interrupt as soon as it is {\em certain} that it has
exceeded its receive capacity. This detection must be performed carefully and
must take into account that two bits beyond the complete transmission may be
temporarily received (as is the case for Node~1 in
Figure~\ref{fig:interrupt}). The receiving state machine should not transition
into {\sc decided\_to\_interrupt} until the \nth{3}~bit of a byte that it does
not have space for. The receiver may wait until the \nth{8}~bit to transition
its state machine to request interrupt, but no later than that such that it is
assured that it will interrupt the transmission before the transmitter
attempts to interrupt to signal end of message even if the receiver is of
lower priority.
% XXX This should be a testbench.. all the specs -> tb's?

\subsection{Clock Speed}
\hl{TODO: What is appropriate specification for clock speeds?} I2C has defined
`speed-classes'; SPI is a free-for-all; UART \& co has bauds...

\subsection{Addressing}
\label{sec:spec-address}
\bus defines two types of addresses: \textit{short} 8-bit addresses and
\textit{full} 32~bit addresses. No short address may begin with {\tt 1111} and
all full addresses must begin with {\tt 1111}. Address assignment is laid out
in~\ref{sec:addressing-short}~\nameref{sec:addressing-short}
and~\ref{sec:addressing-full}~\nameref{sec:addressing-full} respectively.
%
Addresses {\tt 0x00}, {\tt 0x01}, and {\tt 0xffffffff} are reserved. Their
function is detailed in~\ref{sec:control}~\nameref{sec:control}.

\subsection{Unassigned Short Prefix: \texttt{0b1111}}
\label{sec:spec-unassigned-short-prefix}
Out of reset, nodes self-assign a short prefix of {\tt 0b1111}. This short
prefix shall be considered as a sentinel value where short prefixes are
reported, indicating that the node does not currently have a short prefix
assigned.

\subsection{Interrupt Rules}
\label{sec:spec-interrupt}
\bus permits any node to interrupt any message for any reason. To allow for
minimal forward progress, however, \bus requires that nodes permit a minimum
of 32~bits (4~bytes) be transmitted without interruption. An exception is made
for the transmitting node, which is permitted to send messages shorter than
32~bits.

Nodes wishing to interrupt must wait until either 41 or 65 clocks after
Begin~Transmission to interrupt the bus (depending on whether the current
message was addressed to a short address or full address). The interrupting
node must wait until it has latched the \nth{33} bit of data before attempting
to interrupt\footnote{
  If the node instead attempted to interrupt on the \nth{32} bit and was a
  topologically higher-priority node, it would override the transmitter during
  Interrupt priority arbitration, not permitting the transmitter to signal
  End~of~Message.}.

\subsection{Failed Arbitration (Spurious Wakeup)}
\label{sec:spec-spurious}
During normal arbitration, when the arbitration is resolved the control node's
{\tt DIN} line should be low no matter who is participating. If the control
node finds its {\tt DIN} line is high on the arbitration edge, it would
indicate that no one is participating. The control node must treat this as an
error and reset the bus.

The control node {\bf must not} immediately interrupt the bus, however. As
laid out in~\ref{sec:design-power-gated}~\nameref{sec:design-power-gated}, the
arbitration edges may be used as input to node sleep controllers. To avoid
potential issues related to half-waking power gated nodes, the control node
must wait until Begin~Transmission (that is, have generated four clock edges)
before interrupting the bus.

The control bits {\bf must} be driven by the control layer. They {\bf must} be
{\tt 00}, general error.
