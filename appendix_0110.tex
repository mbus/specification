% \svnInfo $Id$

\subsection{MES 0110, ACK 0110, RST 010}

\subsubsection{$1\phi$ Offset}

\begin{figure}
\begin{subfigure}{\textwidth}
    \centering
    \huge {\bf MES 0110, ACK 0110, RST 01011 Broken:}

    \tiny
    \begin{tikztimingtable}[timing/slope=.3,timing/wscale=1.0]
      Din  & L0.8L4{4L}4{4H}2{4L}0.2LLLXXHHH4H    0.5H8{2L}8{2H}8{2L}7{2H}1.5HH \\
      Dout & L0.0L4{4L}4{4H}2{4L}   LLLXXHHH4H    0.5H8{2L}8{2H}8{2L}7{2H}1.5HH \\
      ~~~~ & D{}{16D{Data 0}}{16D{Data 1}}{12D{Data 0}}{44D{Error (Forwarding)}}{28D{Reset}}D
      \\
      \\
      Din  & L3{4L}LLLXXHHH2{4H}HHHXXLLL{4L}{4L}.5L.5HHHH4H    0.8H8{2L}8{2H}8{2L}7{2H}1.2HH \\
      Dout & L3{4L}LLLXXHHH2{4H}HHHXXLLL2{4L}2{4H} HLLL8H4LLH8{2H}8{2L}7{2H}H \\
      ~~~~ & {5D{Data 0}}{32D{End Sequence}}{28D{Acknowledgement Attempt}}{32D{Error (Forwarding)}}{20D{Reset}}D \\
      %\\
      Din  & L0.5L4{4L}4{4H}3{4L}{4H}3.5H HLLL8H4LLH8{2H}8{2L}7{2H}H \\
      Dout & L0.5L4{4L}4{4H}3{4L}{4H}3.5H 8{2L}8{2H}8{2L}7{2H}2.0HH \\
      CLK  & C26{2C}4L15{4C}H \\
      \extracode
        \begin{pgfonlayer}{background}
          \begin{scope}[semitransparent,dashed]
            \vertlines{1,9,...,49}
            \pgfmathparse{\twidth-4}
            \vertlines{57,65,...,\pgfmathresult}
            \vertlines[color=red]{5,13,...,49}
            \vertlines[color=blue]{53,61,...,\pgfmathresult}
            \pgfmathparse{\twidth-1}
            \vertlines{\pgfmathresult}
          \end{scope}
          \begin{scope}[thick]
            \draw[red]   (17,-4.5)  ellipse (2 and 4);
            \draw[green] (33,-4.5)  ellipse (2 and 4);
            \draw[blue]  (47,-13.5) ellipse (3 and 1.25);
          \end{scope}
          \begin{scope}[semitransparent]
            % TX
            \filldraw[yellow]    ( 1,-0.5) rectangle ( 5,-2.5);
            \filldraw[yellow]    (17,-0.5) rectangle (21,-2.5);
            \filldraw[yellow]    (33,-0.5) rectangle (37,-2.5);
            % RX
            \filldraw[yellow]    (37,-8.5) rectangle (41,-10.5);
            \filldraw[yellow]    (45,-8.5) rectangle (49,-10.5);
            \filldraw[yellow]    (57,-8.5) rectangle (65,-10.5);
            % CLK
            \filldraw[yellow]    ( 1,-16.5) rectangle (53,-18.5);
            \filldraw[cyan,opacity=.25] (53,-14.5) rectangle (\twidth-1, -18.5);
          \end{scope}
          \foreach \n [evaluate=\n as \l using int((\n-1)/4)] in {1,5,...,\twidth}
            \draw (\n,-19) -- +(0,-.2)
              node [below,inner sep=2pt] {\scalebox{.75}{\tiny\l}};
        \end{pgfonlayer}
        \begin{scope}
          [font=\sffamily\small,shift={(-3.0em,-0.5)},anchor=east,color=blue]
          \node at (  0,   0) {TX};
          \node at (  0,  -8) {RX};
          \node at (  0, -15) {Ctl};
        \end{scope}
        \begin{scope}
          [font=\sc\tiny,anchor=north,shift={(0,3em)},color=brown]
          \foreach \x [evaluate=\x] in {1,17,...,47}
            \foreach \offset/\l in {0/Drive1,4/Latch1,8/Drive2,12/Latch2}
              \node [rotate=45] at (\x+\offset,0) {\l};
          \node [rotate=45] at (49,0) {Drive1};
          \def\base{57}
          \node [rotate=45] at (\base+0, 0)  {Latch1};
          \node [rotate=45] at (\base+8, 0)  {Drive2};
          \node [rotate=45] at (\base+16, 0) {Latch2};
          \node [rotate=45] at (\base+24, 0) {Drive1};
          \node [rotate=45] at (\base+32, 0) {Latch1};
          \node [rotate=45] at (\base+40, 0) {Reset};
          \node [rotate=45] at (\base+48, 0) {Reset};
          \node [rotate=45] at (\base+56, 0) {Reset};
        \end{scope}
        \begin{scope}
          [font=\bf\tiny,anchor=north,shift={(.2,-3.1em)},color=red]
          \foreach \x [evaluate=\x] in {1,17,...,47}
            \foreach \offset/\l in {0/L2,4/D1,8/L1,12/D2}
              \node [rotate=45] at (\x+\offset,0) {\l};
          \node [rotate=45] at (49,0) {L2};
          \def\base{57}
          \node [rotate=45] at (\base+0, 0)  {D1};
          \node [rotate=45] at (\base+8, 0)  {L1};
          \node [rotate=45] at (\base+16, 0) {D2};
          \node [rotate=45] at (\base+24, 0) {L2};
          \node [rotate=45] at (\base+32, 0) {D1};
          \node [rotate=45] at (\base+40, 0) {L1};
          \node [rotate=45,color=brown] at (\base+48, 0) {R};
          \node [rotate=45,color=brown] at (\base+56, 0) {R};
        \end{scope}
        \begin{scope}
          [font=\sc\tiny,anchor=north,shift={(0,3em)},color=blue]
          \def\base{53}
          \node [rotate=45] at (\base, 0) {Reset-S0-D};
          \node [rotate=45] at (\base+8, 0) {Reset-S0-L};
          \node [rotate=45] at (\base+16, 0) {Reset-S1-D};
          \node [rotate=45] at (\base+24, 0) {Reset-S1-L};
          \node [rotate=45] at (\base+32, 0) {Reset-S0-D};
          \node [rotate=45] at (\base+40, 0) {Reset-S0-L};
          \node [rotate=45] at (\base+48, 0) {Reset-R-D};
          \node [rotate=45] at (\base+56, 0) {Reset-R-L};
          \node [rotate=45,color=black] at (\base+64, 0) {Idle};
        \end{scope}
    \end{tikztimingtable}
    \caption{Transmitter is sending data bits 0, 1, 0 ({\tt 001100}). The
transmitter latches the receiving layer's high bit during {\sc latch2} at time
11, aborts its transmission, and begins forwarding. At time 16, the RX layer
latches a {\tt 0} when it had attempted to drive a {\tt 1}, and aborts the
transmission.}

    \begin{tikztimingtable}[timing/slope=.3]
      Din  & L0.8L4{4L}4{4H}2{4L}0.2LLLXXHHH4H     0.5H8{2L}8{2H}8{2L}7{2H}1.5HH \\
      Dout & L0.0L4{4L}4{4H}2{4L}   LLLXXHHH4U    4U.5L6{2L}8{2H}8{2L}7{2H}1.5HH \\
      ~~~~ & D{}{16D{Data 0}}{16D{Data 1}}{16D{Data 0}}{24D{Data X}}{16D{Error (Forwarding)}}{28D{Reset}}D
      \\
      \\
      Din  & L3{4L}LLLXXHHH2{4H}HHHXXLLL{4L}{4L}.5L.5HHHH4X 4X0.8X 6{2L}    8{2H}8{2L}7{2H}1.2HH \\
      Dout & L3{4L}LLLXXHHH2{4H}HHHXXLLL2{4L}{4H}4X         4X4{2H}2{2L}0.8L8{2H}8{2L}7{2H}1.2HH\\
      ~~~~ & {5D{Data 0}}{32D{End Sequence}}{28D{Acknowledgement}}{32D{Error (Forwarding)}}{20D{Reset}}D \\
      %\\
      Din  & L0.5L4{4L}4{4H}3{4L}{4H}4X   4X4{2H}2{2L}8{2H}8{2L}7{2H}1.5HH  \\
      Dout & L0.5L4{4L}4{4H}3{4L}{4H}3.5H     8{2L}8{2H}8{2L}7{2H}2.0HH \\
      CLK  & C26{2C}4L15{4C}H \\
      \extracode
        \begin{pgfonlayer}{background}
          \begin{scope}[semitransparent,dashed]
            \vertlines{1,9,...,49}
            \pgfmathparse{\twidth-4}
            \vertlines{57,65,...,\pgfmathresult}
            \vertlines[color=red]{5,13,...,49}
            \vertlines[color=blue]{53,61,...,\pgfmathresult}
            \pgfmathparse{\twidth-1}
            \vertlines{\pgfmathresult}
          \end{scope}
          \begin{scope}[thick]
            \draw[red]   (17,-4.5)  ellipse (2 and 4);
            \draw[green] (33,-4.5)  ellipse (2 and 4);
            \draw[blue]  (47,-13.5) ellipse (3 and 1.25);
          \end{scope}
          \begin{scope}[semitransparent]
            % TX
            \filldraw[yellow]    ( 1,-0.5) rectangle ( 5,-2.5);
            \filldraw[yellow]    (17,-0.5) rectangle (21,-2.5);
            \filldraw[yellow]    (33,-0.5) rectangle (37,-2.5);
            \filldraw[yellow]    (49,-0.5) rectangle (57,-2.5);
            % RX
            \filldraw[yellow]    (37,-8.5) rectangle (41,-10.5);
            \filldraw[yellow]    (45,-8.5) rectangle (49,-10.5);
            \filldraw[yellow]    (57,-8.5) rectangle (65,-10.5);
            % CLK
            \filldraw[yellow]    ( 1,-16.5) rectangle (53,-18.5);
            \filldraw[cyan,opacity=.25] (53,-14.5) rectangle (\twidth-1, -18.5);
          \end{scope}
          \foreach \n [evaluate=\n as \l using int((\n-1)/4)] in {1,5,...,\twidth}
            \draw (\n,-19) -- +(0,-.2)
              node [below,inner sep=2pt] {\scalebox{.75}{\tiny\l}};
        \end{pgfonlayer}
        \begin{scope}
          [font=\sffamily\small,shift={(-3.0em,-0.5)},anchor=east,color=blue]
          \node at (  0,   0) {TX};
          \node at (  0,  -8) {RX};
          \node at (  0, -15) {Ctl};
        \end{scope}
        \begin{scope}
          [font=\sc\tiny,anchor=north,shift={(0,3em)},color=brown]
          \foreach \x [evaluate=\x] in {1,17,...,47}
            \foreach \offset/\l in {0/Drive1,4/Latch1,8/Drive2,12/Latch2}
              \node [rotate=45] at (\x+\offset,0) {\l};
          \node [rotate=45] at (49,0) {Drive1};
          \def\base{57}
          \node [rotate=45] at (\base+0, 0)  {Latch1};
          \node [rotate=45] at (\base+8, 0)  {Drive2};
          \node [rotate=45] at (\base+16, 0) {Latch2};
          \node [rotate=45] at (\base+24, 0) {Drive1};
          \node [rotate=45] at (\base+32, 0) {Latch1};
          \node [rotate=45] at (\base+40, 0) {Reset};
          \node [rotate=45] at (\base+48, 0) {Reset};
          \node [rotate=45] at (\base+56, 0) {Reset};
        \end{scope}
        \begin{scope}
          [font=\bf\tiny,anchor=north,shift={(.2,-3.1em)},color=red]
          \foreach \x [evaluate=\x] in {1,17,...,47}
            \foreach \offset/\l in {0/L2,4/D1,8/L1,12/D2}
              \node [rotate=45] at (\x+\offset,0) {\l};
          \node [rotate=45] at (49,0) {L2};
          \def\base{57}
          \node [rotate=45] at (\base+0, 0)  {D1};
          \node [rotate=45] at (\base+8, 0)  {L1};
          \node [rotate=45] at (\base+16, 0) {D2};
          \node [rotate=45] at (\base+24, 0) {L2};
          \node [rotate=45] at (\base+32, 0) {D1};
          \node [rotate=45] at (\base+40, 0) {L1};
          \node [rotate=45,color=brown] at (\base+48, 0) {R};
          \node [rotate=45,color=brown] at (\base+56, 0) {R};
        \end{scope}
        \begin{scope}
          [font=\sc\tiny,anchor=north,shift={(0,3em)},color=blue]
          \def\base{53}
          \node [rotate=45] at (\base, 0) {Reset-S0-D};
          \node [rotate=45] at (\base+8, 0) {Reset-S0-L};
          \node [rotate=45] at (\base+16, 0) {Reset-S1-D};
          \node [rotate=45] at (\base+24, 0) {Reset-S1-L};
          \node [rotate=45] at (\base+32, 0) {Reset-S0-D};
          \node [rotate=45] at (\base+40, 0) {Reset-S0-L};
          \node [rotate=45] at (\base+48, 0) {Reset-R-D};
          \node [rotate=45] at (\base+56, 0) {Reset-R-L};
          \node [rotate=45,color=black] at (\base+64, 0) {Idle};
        \end{scope}
    \end{tikztimingtable}
    \caption{Transmitter is sending data bits 0, 1, 0 ({\tt 001100}). The
transmitter latches {\tt 0} at time 11 and does not yet know there is an
issue. At time 12, so long as the transmitter is sending another full bit
it does not matter whether that bit is a {\tt 0} or {\tt 1} as the control
layer's reset will propagate during the {\sc latch1}, {\sc drive2}, and {\sc
latch2} states.  Shown here is the assumption that the RX node latches the
previous {\tt 1} at time 12. If it had latched {\tt 0} instead it would enter
Reset at time 12.}


    \begin{tikztimingtable}[timing/slope=.3]
      Din  & L0.8L4{4L}4{4H}{4L}{4L}{4H}{4H}{4H}{4H}{4L}{4L}{16L} \\
      Dout & L0.0L4{4L}4{4H}{4L}{4L}{4H}{4H}{4H}{4H}{4L}{4L}{16L}\\
      ~~~~ & D{}{16D{Data 0}}{16D{Data 1}}{8D{MES 0}}{8D{MES 1}}{8D{MES 1}}{8D{MES 0}}{8D{ACK 0}}{8D{NAK}}
      \\
      \\
      Din  & L3{4L}LLLXXHHH2{4H}HHHXXLLL{3L}XX{3H}{4H}{4H}{3H}XX{3L}{4L}{16L} \\
      Dout & L3{4L}LLLXXHHH2{4H}HHHXXLLL{4L}{4H}{4H}{4H}{4H}X{3L}{4L}{16L} \\
      ~~~~ & {5D{Data 0}}{32D{End Sequence}}{32D{Acknowledgement}}D{} \\
      %\\
      Din  & L0.5L4{4L}4{4H}{4L}{4L}{4H}{4H}{4H}{4H}{4L}{4L}{16L} \\
      Dout & L0.5L4{4L}4{4H}{4L}{4L}{4H}{4H}{4H}{4H}{4L}{4L}{16L} \\
      %CLK  & C26{2C}4L15{4C}H \\
      CLK  & C46{2C}4L \\
      \extracode
        \begin{pgfonlayer}{background}
          \begin{scope}[semitransparent,dashed]
            \vertlines{1,9,...,49}
            \pgfmathparse{\twidth-4}
            \vertlines{57,65,...,\pgfmathresult}
            \vertlines[color=red]{5,13,...,49}
            \vertlines[color=blue]{53,61,...,\pgfmathresult}
            \pgfmathparse{\twidth-1}
            \vertlines{\pgfmathresult}
          \end{scope}
          \begin{scope}[semitransparent]
            % TX
            \filldraw[yellow]    ( 1,-0.5) rectangle ( 5,-2.5);
            \filldraw[yellow]    (17,-0.5) rectangle (21,-2.5);
            \filldraw[yellow]    (33,-0.5) rectangle (37,-2.5);
            \filldraw[yellow]    (41,-0.5) rectangle (45,-2.5);
            \filldraw[yellow]    (49,-0.5) rectangle (53,-2.5);
            \filldraw[yellow]    (57,-0.5) rectangle (61,-2.5);
            % RX
            \filldraw[yellow]    (37,-8.5) rectangle (41,-10.5);
            \filldraw[yellow]    (45,-8.5) rectangle (49,-10.5);
            \filldraw[yellow]    (53,-8.5) rectangle (57,-10.5);
            \filldraw[yellow]    (61,-8.5) rectangle (65,-10.5);
          \end{scope}
        \end{pgfonlayer}
        \begin{scope}
          [font=\sffamily\small,shift={(-3.0em,-0.5)},anchor=east,color=blue]
          \node at (  0,   0) {TX};
          \node at (  0,  -8) {RX};
          \node at (  0, -15) {Ctl};
        \end{scope}
        \begin{scope}
          [font=\sc\tiny,anchor=north,shift={(0,3em)},color=brown]
          \foreach \x [evaluate=\x] in {1,17,...,\twidth}
            \foreach \offset/\l in {0/Drive1,4/Latch1,8/Drive2,12/Latch2}
              \node [rotate=45] at (\x+\offset,0) {\l};
        \end{scope}
        \begin{scope}
          [font=\bf\tiny,anchor=north,shift={(.2,-3.1em)},color=red]
          \foreach \x [evaluate=\x] in {1,17,...,\twidth}
            \foreach \offset/\l in {0/L2,4/D1,8/L1,12/D2}
              \node [rotate=45] at (\x+\offset,0) {\l};
        \end{scope}
        \begin{scope}
          [font=\sc\tiny,anchor=north,shift={(0,3em)},color=blue]
          \def\base{53}
        \end{scope}
    \end{tikztimingtable}
    \caption{Transmitting is sending data bits 0, 1, then
Message~End~Sequence ({\tt 00110110})}
\end{subfigure}
\caption{Detail of a $1\phi$TX~0$\rightarrow$1 error. The top labels are the
current global bus state. The red labels above the RX waveform indicate the
local $1\phi$ state machine. {\tt DOUT} lines are colored to indicate when a
node is driving (yellow), otherwise it is forwarding its {\tt DIN} value.
This scheme fails in the third case (TX 0,1,MES) as the RX node can
incorrectly conclude it has ACK'd a message. While this is detectable by
observing that the ACK is not immediately followed by Reset, this is
cumbersome and not very intuitive.
}
\label{fig:reset-1phi-tx-0-1}
\end{figure}

