\section{\proto Programmer's Model}
\label{sec:programmer}

In addition to \proto, \bus also defines a suggested programmer API. While
\proto implementations are not required to adhere to this specification, it is
highly recommended. This specification includes a lightweight software library
built on top of this API.

\begin{figure*}[!h]
\begin{tabular}{l l}
  {\em asynchronous} & Dropped message notification. \\
  {\em asynchronous} & Notification of received message and type. \\
\end{tabular}
\end{figure*}

\subsection{Overview, Memory Map, and Interrupts}
\proto requires a $2^{24}$ region of memory mapped I/O space. By default,
\proto uses {\tt 0xA5000000-0xA6000000}, however this may be changed by
defining the macro {\tt MPQ\_MMIO\_PREFIX} before including {\tt mpq.h}.
The next most significant byte is used to identify the type of request. The
remaining five bytes are defined by each request type:

\begin{tabular}{ c|c|l|l }
  {\tt 0xA50}  & \tt RW & {\em  synchronous} & \nameref{reg-conf} \\
  {\tt 0xA51}  & \tt R~ & {\em  synchronous} & \nameref{reg-int} \\
  {\tt 0xA52}  & \tt ~W & {\em asynchronous} & \nameref{reg-tx-multi} \\
  {\tt 0xA53}  & \tt R~ & {\em  synchronous} & \nameref{reg-tx-result} \\
  {\tt 0xA55}  & \tt RW & {\em  synchronous} & \nameref{reg-rx-multi} \\
  {\tt \vdots} & \tt ~~ &                    & \em Reserved \\
\end{tabular}

In the following sections, there are many reserved bits. These bits shall be
considered RZWI (Read as Zero, Write Ignored). These bits are currently
unused but are reserved for possible future use. Software should write {\tt 0}
to reserved bits to ensure future compatibility.

\subsection{Configurable Parameters}

\subsubsection{Interrupt Configuration [Read/Write]}
\label{reg-conf}
\begin{tabular}{p{9cm} c}
\vspace{-5em}
Depending on the system design, it may be desirable for an interrupt to fire
when a transmission request is completed. Given the very different nature of
long and short requests, they are configured separately.

There is no need to explicitly enable/disable receipt interrupts, as they are
implicitly disabled if all buffer target written bits are set.

&

\begin{tabular}{r l}
  Bits Required & Purpose \{Abbrv\} \\
  \hline
  \hline
  \multicolumn{1}{l}{\em Address} & \\
  8 & \bus Memory Map Location \\
  4 & Command Specifier \\
  \multicolumn{1}{l}{\em Data} & \\
  1 & Long TX Complete Interrupt \{TX L\} \\
  1 & Short TX Complete Interrupt \{TX S\} \\
  1 & Bus dropped an RX~\ref{int-rx-drop} \{RX D\} \\
  1 & Bus received a message~\ref{int-rx-rx} \{RX R\} \\
\end{tabular}

\\

\end{tabular}

\begin{figure}[!h]
\begin{centering}

\begin{bytefield}{32}
  \bitheader{0-31} \\
  \begin{leftwordgroup}{\tt Address}
    \colorbitbox{lightgreen}{8}{0xA5}
    \colorbitbox{lightergreen}{4}{0x0}
    \bitbox{20}{\em Reserved}
  \end{leftwordgroup} \\
\end{bytefield}

\begin{bytefield}{32}
  \bitheader{0-31} \\
  \begin{leftwordgroup}{\tt ~~~Data}
    \bitbox{28}{\em Reserved}
    \bitbox{1}{\begin{sideways}{\tiny RX R}\end{sideways}}
    \bitbox{1}{\begin{sideways}{\tiny RX D}\end{sideways}}
    \bitbox{1}{\begin{sideways}{\tiny TX S}\end{sideways}}
    \bitbox{1}{\begin{sideways}{\tiny TX L}\end{sideways}}
  \end{leftwordgroup} \\
\end{bytefield}

\end{centering}
\end{figure}

\pagebreak

\subsection{TX Transaction Registers}
A properly written program will only issue one transaction at a time---that
is, every request to send a message on the bus will be followed read of the
success of that transaction before issuing another one. If a program does
issue two requests back-to-back, the layer controller should block the second
request until the first is completed. The second request should unblock the
memory bus as soon at the first request is complete and the layer controller
is ready to begin processing the second request. A request for the last
transaction result should reflect the second transaction, blocking if
necessary. There is no method for such an erroneously coded program to check
the return status of the first message.

\subsubsection{Arbitrarily Long Write [Write Only]}
\label{reg-tx-multi}
\begin{tabular}{p{9cm} c}
\vspace{-4.5em}
This transaction requires a buffer to read from and a length. Instead of
duplicating memory units and copying a buffer into the layer controller, the
layer controller instead DMA's directly from memory. This means the layer
controller should have enough local buffer to always stay ahead of the bus
controller and have data prepared in case it loses memory bus arbitration.

For the programmer, this means the region pointed to during this transaction
is immutable. It is undefined what occurs if a write to the transmission
buffer is attempted.

&

\begin{tabular}{r l}
  Bits Required & Purpose \\
  \hline
  \hline
  \multicolumn{1}{l}{\em Address} & \\
  8 & \bus Memory Map Location \\
  4 & Command Specifier \\
  8 & Message Destination Address \\
  8 & Number of {\em words} to send (max 256) \\
  \multicolumn{1}{l}{\em Data} & \\
  32 & Pointer DMA buffer start \\
\end{tabular}

\\
\end{tabular}

\begin{figure}[!h]
\begin{centering}

\begin{bytefield}{32}
  \bitheader{0-31} \\
  \begin{leftwordgroup}{\tt Address}
    \colorbitbox{lightgreen}{8}{0xA5}
    \colorbitbox{lightergreen}{4}{0x2}
    \bitbox{4}{\em Rsvd}
    \bitbox{8}{Length-1}
    \bitbox{8}{\footnotesize Destination Address}
  \end{leftwordgroup} \\
\end{bytefield}

\begin{bytefield}{32}
  \bitheader{0-31} \\
  \begin{leftwordgroup}{\tt ~~~Data}
    \bitbox{32}{Pointer to start of message}
  \end{leftwordgroup} \\
\end{bytefield}

\end{centering}
\end{figure}

\subsubsection{Transaction Result [Read Only]}
\label{reg-tx-result}
\begin{tabular}{p{9cm} c}
\vspace{-4em}
A register that stores the result of the previous transaction request.
``Result'' is defined as the number of bytes successfully sent on the bus
before a reset event occurred and whether the transaction was ACK or NAK'd.
Note that the ACK bit will only be set if the entire transaction completed
successfully and was ACK'd. The length field is primarily useful for a NAK'd
transmission as a {\em hint} to the programmer of what {\em likely} went
wrong. If a message is ACK'd, the length field must match the length of the
request.

The register should block, not returning from the read request until the bus
transaction is complete (though see~\ref{sec:todo-amba} for some concerns). A
read of this register before a transaction has been request will return with a
NAK, the rest of the register contents is undefined.

&

\begin{tabular}{r l}
  Bits Required & Purpose \\
  \hline
  \hline
  \multicolumn{1}{l}{\em Address} & \\
  8 & \bus Memory Map Location \\
  4 & Command Specifier \\
  \multicolumn{1}{l}{\em Data} & \\
  1 & ACK (1) or NAK (0) \\
  8 & Number of {\em words} sent \\
\end{tabular}

\\
\end{tabular}

\begin{figure}[!h]
\begin{centering}

\begin{bytefield}{32}
  \bitheader{0-31} \\
  \begin{leftwordgroup}{\tt Address}
    \colorbitbox{lightgreen}{8}{0xA5}
    \colorbitbox{lightergreen}{4}{0x3}
    \bitbox{20}{\em Reserved}
  \end{leftwordgroup} \\
\end{bytefield}

\begin{bytefield}{32}
  \bitheader{0-31} \\
  \begin{leftwordgroup}{\tt ~~~Data}
    \bitbox{1}{\begin{sideways}{\tiny ACK}\end{sideways}}
    \bitbox{15}{\em Reserved}
    \bitbox{8}{Length-1}
    \bitbox{8}{\em Reserved}
  \end{leftwordgroup} \\
\end{bytefield}

\end{centering}
\end{figure}

\subsection{Transaction RX}

\subsubsection{RX Interrupts}

\paragraph{RX Dropped}
\label{int-rx-drop}
A reception was dropped (bus controller sent reset / did not ACK) by the bus
because there was not enough room to store the message. This interrupt may be
suppressed by~\ref{reg-conf}.

\paragraph{RX Received}
\label{int-rx-rx}
A message has been received. Depending on the expense of
interrupts~(TODO~\ref{sec:todo-interrupts}), this may be a unique interrupt
per RX Buffer or a single shared interrupt with a register to read where a
message came in. This interrupt may be suppressed by~\ref{reg-conf}.
\textbf{Disabling this interrupt will disable message reception.} The bus
controller will NAK even single-word messages if this interrupt is disabled.

\subsubsection{RX Buffer (multi-word) [Read/Write]}
\label{reg-rx-multi}
\begin{tabular}{p{9cm} c}
\vspace{-6em}
Incoming messages need to be stored somewhere, they also may take a non-zero
time to copy out of memory, or in many cases actually copying the message may
be a completely unnecessary and wasteful operation.

The layer controller has \hl{NNNN} configurable message buffer targets. These
targets hold a memory address and a maximum length in words. The target
address must be word-aligned, however the bottom two bits of the address are
reserved for possible future use. Software should take care to write in
word-aligned addresses to this field and to clear the bottom bits when reading
back before use in case they are used by a future implementation. When the
layer controller uses a buffer to store a message, it sets the length of the
received transaction, which marks that buffer as invalid for future use by the
layer controller.

Out of reset, the RX'd message length is non-zero for every target, meaning
the layer is incapable of receiving multi-word messages directly out of reset.
For programs not designed to receive large messages, this will set the M3
system up to automatically NAK long transactions with no programmer
intervention.

&

\begin{tabular}{r l}
  Bits Required & Purpose \\
  \hline
  \hline
  \multicolumn{1}{l}{\em Address} & \\
  8 & \bus Memory Map Location \\
  4 & Command Specifier \\
  2-4 & Buffer Target Index \\
  \multicolumn{1}{l}{\em Data} & \\
  8 & Length of RX'd message \\
  8 & Length of buffer (256 {\em word} max) \\
  16 & Buffer memory address \\
\end{tabular}

\\
\end{tabular}

\begin{figure}[!h]
\begin{centering}

\begin{bytefield}[bitwidth=1em]{32}
  \bitheader{0-31} \\
  \begin{leftwordgroup}{\tt Address}
    \colorbitbox{lightgreen}{8}{0xA5}
    \colorbitbox{lightergreen}{4}{0x5}
    \bitbox{16}{\em Reserved}
    \bitbox{4}{\footnotesize Buffer Index}
  \end{leftwordgroup} \\
\end{bytefield}

\begin{bytefield}[bitwidth=1em]{32}
  \bitheader{0-31} \\
  \begin{leftwordgroup}{\tt ~~~Data}
    \bitbox{8}{\footnotesize Length Received - 1}
    \bitbox{8}{\footnotesize Buffer Length - 1}
    \bitbox{14}{Buffer Address}
    \bitbox{1}{\em R}
    \bitbox{1}{\em R}
  \end{leftwordgroup} \\
\end{bytefield}

\end{centering}
\end{figure}
