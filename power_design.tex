\svnInfo $Id$

\section{Power Design}
\label{sec:power}
The purpose of \bus is to support very low power operation. As such, it is
expected that systems leveraging \bus may need to support power-gating all or
part of the system. In this section, we discuss the requirements to support
power-gated systems, how such systems integrate with \bus, and how
power-oblivious chips can seamlessly interact with hyper power-conscious
chips, promoting interoperability.

\subsection{A Brief Background on Power-Gating}
In the low-power design space, a simple and important concept is the ability
to power-gate, or selectively disable, portions of a system that would
otherwise be idle. By power-gating---removing power from that section of a
chip---, the power consumption of idle silicon goes to zero.

Several challenges with power-gating include: how to preserve state during
idle windows, how to connect power-gated modules to other powered or
power-gated modules, and how to wake and sleep power-gated modules
deterministically. This document does not seek to address all these issues,
%                                              You can get a PhD for that!
rather to demonstrate how \bus can help system designers and the signals that
are ``safe'' to utilize. For a more detailed reference power-gated design with
MBus, consult the {\em \bus~M3~Implementation~Specification}.

In general, sleeping and waking power-gated modules requires a series of
events to occur. In particular, the following signals define common power
control design:

\begin{quote}
\begin{tabular}{r l l c c}
  Signal Name        & Function  & Power-Up & Power-Down \\
  \hline \hline
  {\tt POWER\_ON}    & Controls Power-Gating           & \nth{1} & \nth{2} \\
  {\tt RELEASE\_CLK} & Supply Clock to Internal Logic  & \nth{2} & \nth{2} \\
  {\tt RELEASE\_RST} & (De)Assert Reset                & \nth{3} & \nth{2} \\
  {\tt RELEASE\_ISO} & Electrically Isolate Module I/O & \nth{4} & \nth{1} \\
\end{tabular}
\end{quote}

For the purposes of this document, we are concerned with two modules: (i) The
block that interface with the bus itself---we define this as the {\tt Bus
Controller}---, and (ii) the node that is attached to the bus. With \bus, a
completely power-gated node can seamlessly awaken its {\tt Bus~Controller}
with no special assitance from the sending node or the control node. A {\tt
Bus~Controller} can filter addresses, only waking the entire node for a mesage
destined for that node. Additionally, a power-gated node with a simple
always-on low power interrupt generater can exploit \bus features to generate
the required edges with no specilization or externally synchronized knowlege
of chip status. Finally, devices can reliably detect a shutdown message sent
on the bus and use remaining control edges to shut down both the node and the
node's {\tt Bus~Controller}.

\subsection{Waking the \texttt{Bus~Controller}}
\label{sec:power-bus-controller-wakeup}
Referring to edges from Figure~\ref{fig:arbitration}, edges 1, 2, 3, and~4
provide the required signals. Mapping power edges to \bus protocol edges:

\begin{quote}
\begin{tabular}{r c l}
  Arbitration    & $\rightarrow$ & {\tt POWER\_ON} \\
  Priority Drive & $\rightarrow$ & {\tt RELEASE\_CLK} \\
  Priority Latch & $\rightarrow$ & {\tt RELEASE\_RST} \\
  Drive Bit 0    & $\rightarrow$ & {\tt RELEASE\_ISO} \\
\end{tabular}
\end{quote}

In practice, most {\tt Bus~Controller} implementations will not require the
{\tt RELEASE\_CLK} signal as the \bus clock is (by definition) sufficient for
all bus operations, however it is included in considerations for designs that
may require it. A {\tt Bus~Controller} that is awoken using \bus edges will
find its first rising clock edge to be Latch~Bit~0, the MSB of the address,
and should design state machines appropriately.

\subsubsection{Handling Interrupts During Wakeup}
\label{sec:power-bus-controller-wakeup-int}
By specficiation, Interrupts are not permitted during arbitration. If an
Interrupt occurred, an ignorant {\tt Bus~Controller} would hang, unable to
make forward progress as the remaining edges would have been driven from
clocking the control bits, causing the {\tt Bus~Controller} to interperet
either Latch Control Bit~0 or Latch Control Bit~1 as the MSB of the
destination address. After one or two more cycles, the bus would become idle
while the local {\tt Bus~Controller} waits indefinitely for more bits. This
condition would eventually resovle itself if any other node elected to send a
message but could not be resolved by any other means\footnote{
  Excepting things such as a local timeout and an external reset, but such
  a design is outside the scope of the discussion for a \bus member node.}.

As the {\tt Bus~Controller} was previously powered down, powering it down
again before releasing isolation is by definition a null operation. As
isolation is released on the final falling edge of arbitration and Interrupt
may only occur while the clock is high, a wake-up that is not Interrupted is
safe and cancelling (re-power-gating) a wake-up that is Interrupted is safe.
The challenge is then that the sleep controller module that generates the
power control edges must also be capable of detecting an Interrupt. Whether
this level of robustness is required is left as an implementation decision.

\subsection{Waking the Node}
If the {\tt Bus~Controller}'s address matches the destination address, it must
wake whatever it is attached next up the chain, this means the clockless
{\tt Bus~Controller} module must harvest clock edges from \bus to generate the
power control signals.

One design point explicitly required by \bus is the acknowledgement of
zero-length messages. Depending on application, a node may not require
awakening for a zero-length message. Due to the nature of \bus Interrupt
procedure, however, as many as two bits may be received that will be discarded
(Figure~\ref{fig:interrupt}). A {\tt Bus~Controller} design that attempts to
minimize wakeups should therefore not begin the wakeup process until latching
the {\em \nth{3}} data bit.

\subsubsection{Handling Interrupts During Wakeup}
As the control bits provide ample edges, designers have more options for
handling interrupted wakeup. In particular, the same argument regarding wakeup
cancellation and arbitration from~\ref{sec:power-bus-controller-wakeup-int}
applies: if isolation has not been removed, the node may simply be
re-power-gated without issue.

A possibly simpler implementation can unconditionally complete the wakeup
sequence while indicating that a transaction was started, but failed.

Ultimately, the important consideration is to draw attention to the fact that
an Interrupt {\em may} occur during the {\tt Bus~Controller}'s issuing of
wakeup signals (at any point) and a robust {\tt Bus~Controller} implementation
must consider and handle the cases surrounding Interrupt.
