\section{Scratchpad}
This section contains ideas for future additions to the \bus protocol. Items
in this section are {\bf not} stable and should {\bf not} be implemented in
current \bus designs.

%%%%%%%%%%%%%%%%%%%%%%%%%%%%%%%%%%%%%%%%%%%%%%%%%%%%%%%%%%%%%%%%%%%%%%%%%%%%%%

\paragraph{Broadcast Channel~2: \bus Configuration}
\label{scratch:sec:channel-2}

The purpose of this channel is to configure any \bus parameters. Commands
issued on channel~2 {\bf must} be targeted for the mediator node. By utilizing
a broadcast channel all interested nodes can easily ``subscribe'' to
configuration messages. Using a broadcast channel also permits nodes to
hard-code the address for configuration messages.

The current \bus specification does not define any channel~2 messages.
Possible future messages include:
\begin{itemize}
  \item Maximum message length
  \item Clock speed
\end{itemize}

\paragraph{Broadcast Channel~3: Member Node Events}
\label{scratch:sec:channel-3}

The purpose of this channel is for broadcast dissemination of events that are
specific to one member node.  It also permits the simplification of a member
node design by permitting the member node to hard-code a destination address
for certain classes of messages (e.g. interrupts).

Member node event messages {\bf must} also include a \hlc[lightergreen]{Member
Node Identifier}---the sending node's current short prefix---to disambiguate
member node events. It is possible, and acceptable, for a member node to
generate a member node event message before enumeration has completed, in
which case nodes without a \nameref{sec:addressing-static-short-prefix} send
their current short prefix \nameref{sec:spec-unassigned-short-prefix}.


\subparagraph{Member Node Level Interrupt}
\label{scratch:cmd:level-interrupt}
~

~

\begin{minipage}{\linewidth}
  \begin{varwidth}{.2\linewidth}
    \centering
    \begin{bytefield}{8}
      \bitheader{0-7} \\
      \colorbitbox{lightblue}{4}{0000}
      \colorbitbox{lightcyan}{4}{0011}
    \end{bytefield}
  \end{varwidth}
+
  \begin{varwidth}{.8\linewidth}
    \centering
    \begin{bytefield}[bitwidth=1.25em]{32}
      \bitheader{0-31} \\
      \colorbitbox{lightgreen}{4}{0000}
      \colorbitbox{lightergreen}{4}{\scriptsize Sending Node Short Prefix}
      \bitbox{24}{Interrupt Status Vector}
    \end{bytefield}
  \end{varwidth}
\end{minipage}

~

This message announces that an interrupt has occurred on this member node. The
semantics of a \bus level interrupt dictate that an interrupt must be
explicitly cleared. A member node {\bf must not} generate another level
interrupt message {\em for the same interrupt} until that interrupt is
explicitly cleared. A {\em different} interrupt may occur before the first
interrupt is cleared, in which case the bit vector {\bf must} indicate both
interrupts as active. Member nodes are permitted to batch or delay interrupts
if appropriate, that is, a single level interrupt message may indicate the
multiple interrupts have occurred.

Clearing interrupts is {\bf not} a Channel~3 message, as doing so would
require all member nodes to wake for every Channel~3 message.  As a
consequence, systems with multiple nodes capable of clearing a member node
interrupt should develop another means of coordinating the responsibility. A
member node designed to have multiple potential controlling nodes should
consider defining a broadcast control channel.

\bus defines active to be logical {\tt 1}. Member nodes whose internal
interrupts are active low must invert the signal before broadcasting the
interrupt vector. The remaining 24~bits of this message are defined as a bit
vector representing the current status of 24~interrupts.


\subparagraph{Member Node Edge Interrupt}
\label{scratch:cmd:edge-interrupt}
~

~

\begin{minipage}{\linewidth}
  \begin{varwidth}{.2\linewidth}
    \centering
    \begin{bytefield}{8}
      \bitheader{0-7} \\
      \colorbitbox{lightblue}{4}{0000}
      \colorbitbox{lightcyan}{4}{0011}
    \end{bytefield}
  \end{varwidth}
+
  \begin{varwidth}{.8\linewidth}
    \centering
    \begin{bytefield}[bitwidth=1.25em]{32}
      \bitheader{0-31} \\
      \colorbitbox{lightgreen}{4}{0001}
      \colorbitbox{lightergreen}{4}{\scriptsize Sending Node Short Prefix}
      \bitbox{24}{Interrupt Status Vector}
    \end{bytefield}
  \end{varwidth}
\end{minipage}

~

This message announces that an interrupt has occurred on this member node. The
semantics of a \bus edge interrupt dictate that the interrupt is implicitly
cleared. That is, after the broadcast message announcing the interrupt has
been sent the same interrupt is eligible to fire again. Member nodes are
permitted to batch or delay interrupts if appropriate, that is, a single edge
interrupt message may indicate multiple interrupts have occurred.

\bus defines active to be logical {\tt 1}. Member nodes whose internal
interrupts are active low must invert the signal before broadcasting the
interrupt vector. The remaining 24~bits of this message are defined as a bit
vector representing the current status of 24~interrupts.



%%%%%%%%%%%%%%%%%%%%%%%%%%%%%%%%%%%%%%%%%%%%%%%%%%%%%%%%%%%%%%%%%%%%%%%%%%%%%%


\subparagraph{Member Node Set Memory Stream Address LSB}
\label{scratch:cmd:conf-mem-stream-lsb}
~

~

\begin{minipage}{\linewidth}
  \begin{varwidth}{.2\linewidth}
    \centering
    \begin{bytefield}{8}
      \bitheader{0-7} \\
      \colorbitbox{lightblue}{4}{0000}
      \colorbitbox{lightcyan}{4}{0011}
    \end{bytefield}
  \end{varwidth}
+
  \begin{varwidth}{.8\linewidth}
    \centering
    \begin{bytefield}[bitwidth=1.25em]{32}
      \bitheader{0-31} \\
      \colorbitbox{lightgreen}{4}{0010}
      \colorbitbox{lightergreen}{4}{\scriptsize Sending Node Short Prefix}
      \bitbox{4}{\scriptsize Target Node Short Prefix}
      \bitbox{1}{\begin{sideways}{\tiny QRY}\end{sideways}}
      \bitbox{2}{\footnotesize Rsvd}
      \bitbox{1}{\begin{sideways}{\tiny CL T}\end{sideways}}
      \bitbox{16}{Memory Stream Address [15:0]}
    \end{bytefield}
  \end{varwidth}
\end{minipage}

~

This message sets the lower halfword of the \nameref{cmd:mem-stream-write}
destination address for the target node specified in bits 23-20. If the {\tt
CL~T} bit is high, the top halfword of the destination address is cleared (set
to zero).

If the {\tt QRY} (query) bit is set, the {\tt CL~B} and address are ignored.
Instead, the node generates a \nameref{cmd:conf-mem-strem-qresp} with the
Queried Parameter field set to {\tt 0011}.

\subparagraph{Member Node Set Memory Stream Address MSB}
\label{scratch:cmd:conf-mem-stream-msb}
~

~

\begin{minipage}{\linewidth}
  \begin{varwidth}{.2\linewidth}
    \centering
    \begin{bytefield}{8}
      \bitheader{0-7} \\
      \colorbitbox{lightblue}{4}{0000}
      \colorbitbox{lightcyan}{4}{0011}
    \end{bytefield}
  \end{varwidth}
+
  \begin{varwidth}{.8\linewidth}
    \centering
    \begin{bytefield}[bitwidth=1.25em]{32}
      \bitheader{0-31} \\
      \colorbitbox{lightgreen}{4}{0011}
      \colorbitbox{lightergreen}{4}{\scriptsize Sending Node Short Prefix}
      \bitbox{4}{\scriptsize Target Node Short Prefix}
      \bitbox{1}{\begin{sideways}{\tiny QRY}\end{sideways}}
      \bitbox{2}{\footnotesize Rsvd}
      \bitbox{1}{\begin{sideways}{\tiny CL B}\end{sideways}}
      \bitbox{16}{Memory Stream Address [31:16]}
    \end{bytefield}
  \end{varwidth}
\end{minipage}

~

This message sets the upper halfword of the \nameref{cmd:mem-stream-write}
destination address for the target node specified in bits 23-20. If the {\tt
CL~B} bit is high, the bottom halfword of the destination address is cleared
(set to zero).

If the {\tt QRY} (query) bit is set, the {\tt CL~B} and address are ignored.
Instead, the node generates a \nameref{cmd:conf-mem-strem-qresp} with the
Queried Parameter field set to {\tt 0011}.


\subparagraph{Member Node Set Memory Stream Length}
\label{scratch:cmd:conf-mem-stream-len}
~

~

\begin{minipage}{\linewidth}
  \begin{varwidth}{.2\linewidth}
    \centering
    \begin{bytefield}{8}
      \bitheader{0-7} \\
      \colorbitbox{lightblue}{4}{0000}
      \colorbitbox{lightcyan}{4}{0011}
    \end{bytefield}
  \end{varwidth}
+
  \begin{varwidth}{.8\linewidth}
    \centering
    \begin{bytefield}[bitwidth=1.25em]{32}
      \bitheader{0-31} \\
      \colorbitbox{lightgreen}{4}{0100}
      \colorbitbox{lightergreen}{4}{\scriptsize Sending Node Short Prefix}
      \bitbox{4}{\scriptsize Target Node Short Prefix}
      \bitbox{1}{\begin{sideways}{\tiny QRY}\end{sideways}}
      \bitbox{2}{\footnotesize Rsvd}
      \bitbox{1}{\begin{sideways}{\tiny EN}\end{sideways}}
      \bitbox{16}{Length-1}
    \end{bytefield}
  \end{varwidth}
\end{minipage}

~

\hl{TODO: Message}


\subparagraph{Member Node Query Memory Stream Response}
\label{scratch:cmd:conf-mem-stream-qresp}
~

~

\begin{minipage}{\linewidth}
  \begin{varwidth}{.2\linewidth}
    \centering
    \begin{bytefield}{8}
      \bitheader{0-7} \\
      \colorbitbox{lightblue}{4}{0000}
      \colorbitbox{lightcyan}{4}{0011}
    \end{bytefield}
  \end{varwidth}
+
  \begin{varwidth}{.8\linewidth}
    \centering
    \begin{bytefield}[bitwidth=1.25em]{32}
      \bitheader{0-31} \\
      \colorbitbox{lightgreen}{4}{0101}
      \colorbitbox{lightergreen}{4}{\scriptsize Sending Node Short Prefix}
      \bitbox{4}{\scriptsize Query Parameter}
      \bitbox{1}{\begin{sideways}{\tiny RSV}\end{sideways}}
        \bitbox{19}{Queried Parameter}
    \end{bytefield}
  \end{varwidth}
\end{minipage}

~

\hl{TODO: Message}

%%%%%%%%%%%%%%%%%%%%%%%%%%%%%%%%%%%%%%%%%%%%%%%%%%%%%%%%%%%%%%%%%%%%%%%%%%%%%%


\paragraph{Broadcast Channel~7: Data}
\label{scratch:scratch:sec:channel-7}
Channel~7 is used to send arbitrary data to every addressable entity on the
bus. \bus places no further restrictions or structure on channel~7 messages.
Channel~7 is intended for use by more flexible software nodes, though any node
may listen or transmit on channel~7.

%%%%%%%%%%%%%%%%%%%%%%%%%%%%%%%%%%%%%%%%%%%%%%%%%%%%%%%%%%%%%%%%%%%%%%%%%%%%%%

    \begin{tabular}{r|r|l}
      {\tt LU} & Resolution & Range \\
      \hline
      {\tt 00} & 1~word     & 1--64~words (4--256~B) \\
      {\tt 01} & 16~words   & 16--1K~words (64--4K~B) \\
      {\tt 10} & 256~words  & 256--16K~words (1K--64K~B) \\
      {\tt 11} & 4K~words   & 4K--256K~words (16K--1M~B) \\
    \end{tabular}

\subsubsection{Memory Stream Read, Configure, or Alert}
\label{scratch:cmd:mem-stream-multi}

\begin{bytefield}{9}
  \bitheader{0-7} \\
  \bitbox[rbt]{5}{{\dots}Prefix}
  \bitbox{4}{0100} \\
  \bitbox[t]{9}{\hspace{-.5em}$\underbrace{\hspace{9.5em}}_{\text{\small \bus Address}}$}
\end{bytefield}
~
\begin{bytefield}[bitwidth=.5em]{64}
%  \bitheader{0,31} \\
  \bitbox[]{1}{\tiny ~ \\ ~ \\ 31}
  \bitbox[]{5}{}
  \bitbox[]{2}{\tiny ~ \\ ~ \\ 24}
  \bitbox[]{1}{\tiny ~ \\ ~ \\ \begin{sideways}{\tiny 23~}\end{sideways}}
  \bitbox[]{1}{\tiny ~ \\ ~ \\ \begin{sideways}{\tiny 22~}\end{sideways}}
  \bitbox[]{1}{\tiny ~ \\ ~ \\ \begin{sideways}{\tiny 21~}\end{sideways}}
  \bitbox[]{1}{\tiny ~ \\ ~ \\ \begin{sideways}{\tiny 20~}\end{sideways}}
  \bitbox[]{2}{\tiny ~ \\ ~ \\ 19}
  \bitbox[]{1}{\tiny ~ \\ ~ \\ \tt \bf \dots}
  \bitbox[]{16}{}
  \bitbox[]{1}{\tiny ~ \\ ~ \\ 0}
  \bitbox[]{2}{\tiny ~ \\ ~ \\ 31}
  \bitbox[]{1}{\tiny ~ \\ ~ \\ \tt \bf \dots}
  \bitbox[]{26}{}
  \bitbox[]{1}{\tiny ~ \\ ~ \\ 2}
  \bitbox[]{1}{\tiny ~ \\ ~ \\ 1}
  \bitbox[]{1}{\tiny ~ \\ ~ \\ 0}
  \\
  \bitbox{8}{\tiny \bus Reply Address}
  \bitbox{1}{\footnotesize 0}
  \colorbitbox{lightgray}{1}{\footnotesize \tt Z}
  \colorbitbox{lightgray}{1}{\footnotesize \tt Z}
  \colorbitbox{lightgray}{1}{\footnotesize \tt Z}
  \bitbox{20}{Length-1}
  \bitbox{30}{Read Start Address}
  \colorbitbox{lightgray}{1}{\footnotesize \tt Z}
  \colorbitbox{lightgray}{1}{\footnotesize \tt Z}
  \bitbox[]{10}{\raggedright ~$\leftarrow$ Read}
  \\
  \bitbox{8}{\tiny Alert \bus Address}
  \bitbox{1}{\footnotesize 1}
  \bitbox{2}{\tiny 01 10 11}
  \bitbox{1}{\begin{sideways}{\tiny DBLB}\end{sideways}}
  \bitbox{20}{Buffer Length-1}
  \bitbox{30}{Write Start Address}
  \colorbitbox{lightgray}{1}{\footnotesize \tt Z}
  \colorbitbox{lightgray}{1}{\footnotesize \tt Z}
  \bitbox[]{12}{\raggedright ~$\leftarrow$ Configure}
  \\
  \bitbox{8}{\tiny Source \bus Address}
  \bitbox{1}{\footnotesize 1}
  \bitbox{2}{\footnotesize 00}
  \colorbitbox{lightgray}{1}{\footnotesize \tt Z}
  \bitbox{20}{Valid Data Length-1}
  \bitbox{30}{Valid Data Start Address}
  \colorbitbox{lightgray}{1}{\footnotesize \tt Z}
  \colorbitbox{lightgray}{1}{\footnotesize \tt Z}
  \bitbox[]{10}{\raggedright ~$\leftarrow$ Alert}
  \\
  \bitbox[t]{64}{$\underbrace{\hspace{30em}}_{\text{\small \bus Data}}$}
\end{bytefield}

Three commands share the {\tt 0100} FU\_ID. Bits~21--23 in the first word
determine whether the command is a read, configure, or alert. A stream read
generates a stream write. A stream configure is an isolated command with no
response. A stream alert is generated in response to a stream write.

\paragraph{Read (Bit 23 == 0):}
\label{scratch:cmd:mem-stream-multi-read}
The first word indicates the \bus address to reply to and the length of the
requested read in words less one.
The second word received is the address in memory to read from. The bottom two
bits of the address field are reserved and {\bf must} be transmitted as 0.

The response is sent immediately and the message is formatted exactly as the
\nameref{cmd:mem-stream-write} command, that is a series of sequential 32~bit
data fields.

\subparagraph{Overflow:} If the starting address field and subsequent length
exceed the memory space, that is a request for address {\tt 0x100000000} would
have been made during the response, the layer controller wraps and continues
sending from address {\tt 0x00000000}.
{\em Tests: \ref{test:mem-stream-overflow}.}

\paragraph{Interjection Semantics:} If the reply is interjected, the
transaction is aborted and is {\bf not} retried.

\paragraph{Configure (Bit 23 == 1, Bits 22-21 in (01,10,11):}
\label{scratch:cmd:mem-stream-multi-conf}
Bits~22--21 specify which streaming channel is currently being configured.
The first byte holds the \bus address that an Alert message will be sent to
when this streaming channel's buffer is full.
Bits~19--0 specify the maximum length of the buffer.
The second word specifies the address in memory to begin writing to.
Bit~20 {\tt DBLB} specifies whether double-buffering mode is active. Without
double-buffering, an Alert is generated once the buffer is full and the buffer
cannot be written into again until another Configure is received. With
double-buffering, an Alert is generated once halfway through the buffer and
again at the end of the buffer. At the end of the buffer, the address resets
to the beginning of the buffer and continues to accept writes.

\paragraph{Alert (Bit 23 == 1, Bits 22-21 == 00):}
\label{scratch:cmd:mem-stream-multi-alert}
The first byte holds the address of the node generating the Alert.
The second word holds the start of the buffer in the local address space and
the end of the first word specifies the length of the buffer that is valid.


%%%%%%%%%%%%%%%%%%%%%%%%%%%%%%%%%%%%%%%%%%%%%%%%%%%%%%%%%%%%%%%%%%%%%%%%%%%%%%

\begin{tabular}{l l}
  {\em asynchronous} & Send single word message to address. \\
  {\em  synchronous} & Retrieve received single-word message. \\
\end{tabular}

\begin{figure}[!h]
\begin{tabular}{ c|c|l }
  {\tt 0xA51} & \tt ~W & \nameref{reg-tx-single} \\
  {\tt 0xA54} & \tt R~ & \nameref{reg-rx-single} \\
\end{tabular}
\end{figure}

\subsubsection{Single Word Write [Write Only]}
\label{scratch:reg-tx-single}
\begin{tabular}{p{9cm} c}
\vspace{-4em}
As a small optimization for the simple, single-word case a single memory write
can express all of the needed information without requiring the layer
controller to issue another request to memory to retrieve data.

As there is no hardware retry primitive, the software is still obligated to
issue a read to determine if the message send was successful. For similar
reasons, this message is an asynchronous message and should return control to
the processor as soon as the layer controller accepts the request, not waiting
for the bus controller to send the message.

&

\begin{tabular}{r l}
  Bits Required & Purpose \\
  \hline
  \hline
  \multicolumn{1}{l}{\em Address} & \\
  8 & \bus Memory Map Location \\
  4 & Command Specifier \\
  8 & Message Destination Address \\
  \multicolumn{1}{l}{\em Data} & \\
  32 & Data to send on \bus \\
\end{tabular}

\\
\end{tabular}

\begin{figure}[!h]
\begin{centering}

\begin{bytefield}{32}
  \bitheader{0-31} \\
  \begin{leftwordgroup}{\tt Address}
    \colorbitbox{lightgreen}{8}{0xA5}
    \colorbitbox{lightergreen}{4}{0x1}
    \bitbox{12}{\em Reserved}
    \bitbox{8}{\footnotesize Destination Address}
  \end{leftwordgroup} \\
\end{bytefield}

\begin{bytefield}{32}
  \bitheader{0-31} \\
  \begin{leftwordgroup}{\tt ~~~Data}
    \bitbox{32}{Data to send on \bus}
  \end{leftwordgroup} \\
\end{bytefield}

\end{centering}
\end{figure}

\subsubsection{RX Word (single-word) [Read Only]}
\label{scratch:reg-rx-single}
\begin{tabular}{p{9cm} c}
\vspace{-4em}
Similar to TX, a shorter path to receive a single word message. If another
message is received prior to this message being read out, the new message
should be NAK'd.

The layer controller may include a small internal queue of received messages.
In that case, the RX~Received interrupt~(\ref{int-rx-rx}) should remain
asserted until all messages have been read out. This will ensure software
cannot disable reception until all messages have been received.

&

\begin{tabular}{r l}
  Bits Required & Purpose \\
  \hline
  \hline
  \multicolumn{1}{l}{\em Address} & \\
  8 & \bus Memory Map Location \\
  4 & Command Specifier \\
  \multicolumn{1}{l}{\em Data} & \\
  32 & Data received \\
\end{tabular}

\\
\end{tabular}

\begin{figure}[!h]
\begin{centering}

\begin{bytefield}{32}
  \bitheader{0-31} \\
  \begin{leftwordgroup}{\tt Address}
    \colorbitbox{lightgreen}{8}{0xA5}
    \colorbitbox{lightergreen}{4}{0x4}
    \bitbox{20}{\em Reserved}
  \end{leftwordgroup} \\
\end{bytefield}

\begin{bytefield}{32}
  \bitheader{0-31} \\
  \begin{leftwordgroup}{\tt ~~~Data}
    \bitbox{32}{Data Received}
  \end{leftwordgroup} \\
\end{bytefield}

\end{centering}
\end{figure}

%%%%%%%%%%%%%%%%%%%%%%%%%%%%%%%%%%%%%%%%%%%%%%%%%%%%%%%%%%%%%%%%%%%%%%%%%%%%%%

%% Old Injection Method
\begin{comment}
For a packaged system using \bus, simply providing an external {\tt DIN} and
{\tt DOUT} that is normally jumpered, plus access to the {\tt CLK} line, is
sufficient to allow transient bus elements (such as a programmer, debugger).

For completely contained systems, it is desirable to also define a bus
injection methodology that can be probed onto a chip, as well as a passive
listening method. \bus defines two additional, optional pads that may be added
to any node:

\begin{itemize}
  \item {\tt DSNOOP} -- Snoop, Data out
  \item {\tt MODE}   -- Connected?, Data in
\end{itemize}

The {\tt MODE} pad is connected to a weak pull-down resistor. During device
power-on, if the {\tt MODE} pad is found to be high, an external device is
assumed to be connected. In this case, the {\tt DIN} pad is routed to the {\tt
DSNOOP} pad and the {\tt MODE} pad is routed to the chip-internal {\tt DIN}
signal. Otherwise the {\tt MODE} pad is ignored and the chip {\tt DIN} pad
routes to both the chip-internal {\tt DIN} signal and the {\tt DSNOOP} pad.
These functions are determined during Power-On-Reset and remain fixed until a
future chip-Reset.
\end{comment}
%% END Old Injection Method

Thought from \nameref{sec:bus-return-idle}:
\begin{quote}
\textit{M3 Implementation Note:} (Referring to
\cref{fig:int}) A broadcast sleep message cannot be considered a
sleep until time~18 when the transmitter asserts that all the sent bits were
desired to be sent. This leaves edges at time~20 and time~22 as the required
two edges to power gate the layer.

This could be complicated, however, if a node elects (for some reason) to send
a response to the broadcast sleep message. The sleep controller will still
have four edges (Arbitration, Priority~Drive, Priority~Latch,
Begin~Transmission) to wake its bus controller, but the timing between arming
its {\tt CLK\_IN} fall detector and the mediator node pulling {\tt CLK\_IN} low
in response could cause the sleep controller to miss wakeup.
\end{quote}
