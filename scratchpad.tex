\section{Scratchpad}
This section contains ideas for future additions to the \bus protocol. Items
in this section are {\bf not} stable and should {\bf not} be implemented in
current \bus designs.

\paragraph{Broadcast Channel~2: \bus Configuration}
\label{sec:channel-2}

The purpose of this channel is to configure any \bus parameters. Commands
issued on channel~2 {\bf must} be targeted for the mediator node. By utilizing
a broadcast channel all interested nodes can easily ``subscribe'' to
configuration messages. Using a broadcast channel also permits nodes to
hard-code the address for configuration messages.

The current \bus specification does not define any channel~2 messages.
Possible future messages include:
\begin{itemize}
  \item Maximum message length
  \item Clock speed
\end{itemize}

\paragraph{Broadcast Channel~3: Member Node Events}
\label{sec:channel-3}

The purpose of this channel is for broadcast dissemination of events that are
specific to one member node.  It also permits the simplification of a member
node design by permitting the member node to hard-code a destination address
for certain classes of messages (e.g. interrupts).

Member node event messages {\bf must} also include a \hlc[lightergreen]{Member
Node Identifier}---the sending node's current short prefix---to disambiguate
member node events. It is possible, and acceptable, for a member node to
generate a member node event message before enumeration has completed, in
which case nodes without a \nameref{sec:addressing-static-short-prefix} send
their current short prefix \nameref{sec:spec-unassigned-short-prefix}.


\subparagraph{Member Node Level Interrupt}
\label{cmd:level-interrupt}
~

~

\begin{minipage}{\linewidth}
  \begin{varwidth}{.2\linewidth}
    \centering
    \begin{bytefield}{8}
      \bitheader{0-7} \\
      \colorbitbox{lightblue}{4}{0000}
      \colorbitbox{lightcyan}{4}{0011}
    \end{bytefield}
  \end{varwidth}
+
  \begin{varwidth}{.8\linewidth}
    \centering
    \begin{bytefield}[bitwidth=1.25em]{32}
      \bitheader{0-31} \\
      \colorbitbox{lightgreen}{4}{0000}
      \colorbitbox{lightergreen}{4}{\scriptsize Sending Node Short Prefix}
      \bitbox{24}{Interrupt Status Vector}
    \end{bytefield}
  \end{varwidth}
\end{minipage}

~

This message announces that an interrupt has occurred on this member node. The
semantics of a \bus level interrupt dictate that an interrupt must be
explicitly cleared. A member node {\bf must not} generate another level
interrupt message {\em for the same interrupt} until that interrupt is
explicitly cleared. A {\em different} interrupt may occur before the first
interrupt is cleared, in which case the bit vector {\bf must} indicate both
interrupts as active. Member nodes are permitted to batch or delay interrupts
if appropriate, that is, a single level interrupt message may indicate the
multiple interrupts have occurred.

Clearing interrupts is {\bf not} a Channel~3 message, as doing so would
require all member nodes to wake for every Channel~3 message.  As a
consequence, systems with multiple nodes capable of clearing a member node
interrupt should develop another means of coordinating the responsibility. A
member node designed to have multiple potential controlling nodes should
consider defining a broadcast control channel.

\bus defines active to be logical {\tt 1}. Member nodes whose internal
interrupts are active low must invert the signal before broadcasting the
interrupt vector. The remaining 24~bits of this message are defined as a bit
vector representing the current status of 24~interrupts.


\subparagraph{Member Node Edge Interrupt}
\label{cmd:edge-interrupt}
~

~

\begin{minipage}{\linewidth}
  \begin{varwidth}{.2\linewidth}
    \centering
    \begin{bytefield}{8}
      \bitheader{0-7} \\
      \colorbitbox{lightblue}{4}{0000}
      \colorbitbox{lightcyan}{4}{0011}
    \end{bytefield}
  \end{varwidth}
+
  \begin{varwidth}{.8\linewidth}
    \centering
    \begin{bytefield}[bitwidth=1.25em]{32}
      \bitheader{0-31} \\
      \colorbitbox{lightgreen}{4}{0001}
      \colorbitbox{lightergreen}{4}{\scriptsize Sending Node Short Prefix}
      \bitbox{24}{Interrupt Status Vector}
    \end{bytefield}
  \end{varwidth}
\end{minipage}

~

This message announces that an interrupt has occurred on this member node. The
semantics of a \bus edge interrupt dictate that the interrupt is implicitly
cleared. That is, after the broadcast message announcing the interrupt has
been sent the same interrupt is eligible to fire again. Member nodes are
permitted to batch or delay interrupts if appropriate, that is, a single edge
interrupt message may indicate multiple interrupts have occurred.

\bus defines active to be logical {\tt 1}. Member nodes whose internal
interrupts are active low must invert the signal before broadcasting the
interrupt vector. The remaining 24~bits of this message are defined as a bit
vector representing the current status of 24~interrupts.




\subparagraph{Member Node Set Memory Stream Address LSB}
\label{cmd:conf-mem-stream-lsb}
~

~

\begin{minipage}{\linewidth}
  \begin{varwidth}{.2\linewidth}
    \centering
    \begin{bytefield}{8}
      \bitheader{0-7} \\
      \colorbitbox{lightblue}{4}{0000}
      \colorbitbox{lightcyan}{4}{0011}
    \end{bytefield}
  \end{varwidth}
+
  \begin{varwidth}{.8\linewidth}
    \centering
    \begin{bytefield}[bitwidth=1.25em]{32}
      \bitheader{0-31} \\
      \colorbitbox{lightgreen}{4}{0010}
      \colorbitbox{lightergreen}{4}{\scriptsize Sending Node Short Prefix}
      \bitbox{4}{\scriptsize Target Node Short Prefix}
      \bitbox{1}{\begin{sideways}{\tiny QRY}\end{sideways}}
      \bitbox{2}{\footnotesize Rsvd}
      \bitbox{1}{\begin{sideways}{\tiny CL T}\end{sideways}}
      \bitbox{16}{Memory Stream Address [15:0]}
    \end{bytefield}
  \end{varwidth}
\end{minipage}

~

This message sets the lower halfword of the \nameref{cmd:mem-stream-write}
destination address for the target node specified in bits 23-20. If the {\tt
CL~T} bit is high, the top halfword of the destination address is cleared (set
to zero).

If the {\tt QRY} (query) bit is set, the {\tt CL~B} and address are ignored.
Instead, the node generates a \nameref{cmd:conf-mem-strem-qresp} with the
Queried Parameter field set to {\tt 0011}.

\subparagraph{Member Node Set Memory Stream Address MSB}
\label{cmd:conf-mem-stream-msb}
~

~

\begin{minipage}{\linewidth}
  \begin{varwidth}{.2\linewidth}
    \centering
    \begin{bytefield}{8}
      \bitheader{0-7} \\
      \colorbitbox{lightblue}{4}{0000}
      \colorbitbox{lightcyan}{4}{0011}
    \end{bytefield}
  \end{varwidth}
+
  \begin{varwidth}{.8\linewidth}
    \centering
    \begin{bytefield}[bitwidth=1.25em]{32}
      \bitheader{0-31} \\
      \colorbitbox{lightgreen}{4}{0011}
      \colorbitbox{lightergreen}{4}{\scriptsize Sending Node Short Prefix}
      \bitbox{4}{\scriptsize Target Node Short Prefix}
      \bitbox{1}{\begin{sideways}{\tiny QRY}\end{sideways}}
      \bitbox{2}{\footnotesize Rsvd}
      \bitbox{1}{\begin{sideways}{\tiny CL B}\end{sideways}}
      \bitbox{16}{Memory Stream Address [31:16]}
    \end{bytefield}
  \end{varwidth}
\end{minipage}

~

This message sets the upper halfword of the \nameref{cmd:mem-stream-write}
destination address for the target node specified in bits 23-20. If the {\tt
CL~B} bit is high, the bottom halfword of the destination address is cleared
(set to zero).

If the {\tt QRY} (query) bit is set, the {\tt CL~B} and address are ignored.
Instead, the node generates a \nameref{cmd:conf-mem-strem-qresp} with the
Queried Parameter field set to {\tt 0011}.


\subparagraph{Member Node Set Memory Stream Length}
\label{cmd:conf-mem-stream-len}
~

~

\begin{minipage}{\linewidth}
  \begin{varwidth}{.2\linewidth}
    \centering
    \begin{bytefield}{8}
      \bitheader{0-7} \\
      \colorbitbox{lightblue}{4}{0000}
      \colorbitbox{lightcyan}{4}{0011}
    \end{bytefield}
  \end{varwidth}
+
  \begin{varwidth}{.8\linewidth}
    \centering
    \begin{bytefield}[bitwidth=1.25em]{32}
      \bitheader{0-31} \\
      \colorbitbox{lightgreen}{4}{0100}
      \colorbitbox{lightergreen}{4}{\scriptsize Sending Node Short Prefix}
      \bitbox{4}{\scriptsize Target Node Short Prefix}
      \bitbox{1}{\begin{sideways}{\tiny QRY}\end{sideways}}
      \bitbox{2}{\footnotesize Rsvd}
      \bitbox{1}{\begin{sideways}{\tiny EN}\end{sideways}}
      \bitbox{16}{Length-1}
    \end{bytefield}
  \end{varwidth}
\end{minipage}

~

\hl{TODO: Message}


\subparagraph{Member Node Query Memory Stream Response}
\label{cmd:conf-mem-stream-qresp}
~

~

\begin{minipage}{\linewidth}
  \begin{varwidth}{.2\linewidth}
    \centering
    \begin{bytefield}{8}
      \bitheader{0-7} \\
      \colorbitbox{lightblue}{4}{0000}
      \colorbitbox{lightcyan}{4}{0011}
    \end{bytefield}
  \end{varwidth}
+
  \begin{varwidth}{.8\linewidth}
    \centering
    \begin{bytefield}[bitwidth=1.25em]{32}
      \bitheader{0-31} \\
      \colorbitbox{lightgreen}{4}{0101}
      \colorbitbox{lightergreen}{4}{\scriptsize Sending Node Short Prefix}
      \bitbox{4}{\scriptsize Query Parameter}
      \bitbox{1}{\begin{sideways}{\tiny RSV}\end{sideways}}
        \bitbox{19}{Queried Parameter}
    \end{bytefield}
  \end{varwidth}
\end{minipage}

~

\hl{TODO: Message}


\paragraph{Broadcast Channel~7: Data}
\label{sec:channel-7}
Channel~7 is used to send arbitrary data to every addressable entity on the
bus. \bus places no further restrictions or structure on channel~7 messages.
Channel~7 is intended for use by more flexible software nodes, though any node
may listen or transmit on channel~7.


%% Old Injection Method
\begin{comment}
For a packaged system using \bus, simply providing an external {\tt DIN} and
{\tt DOUT} that is normally jumpered, plus access to the {\tt CLK} line, is
sufficient to allow transient bus elements (such as a programmer, debugger).

For completely contained systems, it is desirable to also define a bus
injection methodology that can be probed onto a chip, as well as a passive
listening method. \bus defines two additional, optional pads that may be added
to any node:

\begin{itemize}
  \item {\tt DSNOOP} -- Snoop, Data out
  \item {\tt MODE}   -- Connected?, Data in
\end{itemize}

The {\tt MODE} pad is connected to a weak pull-down resistor. During device
power-on, if the {\tt MODE} pad is found to be high, an external device is
assumed to be connected. In this case, the {\tt DIN} pad is routed to the {\tt
DSNOOP} pad and the {\tt MODE} pad is routed to the chip-internal {\tt DIN}
signal. Otherwise the {\tt MODE} pad is ignored and the chip {\tt DIN} pad
routes to both the chip-internal {\tt DIN} signal and the {\tt DSNOOP} pad.
These functions are determined during Power-On-Reset and remain fixed until a
future chip-Reset.
\end{comment}
%% END Old Injection Method
