\svnInfo $Id$

\section{ToDo}
\label{sec:todo}

\subsection{Future Extensions}
\label{sec:todo-extensions}
There are properties that would be nice to have, but introduce greater
complexity. While they are not in the current design, some methods for
designing them are considered here such that they are still feasible for
future implementation in a backwards-compatible manner.

\subsubsection{Resuming Interrupted Transfers}
\label{sec:todo-extensions-resume}
Consider a separate `bulk-transfer' address such that nodes which support
resumable transfers have two addresses (and a corresponding broadcast query
for such addresses).

A bulk transfer always begins with the sending node's full address both as
unique token and so the bulk protocol can respond if needed. Transfers begin
with both ends' counters synchronized at 0~bits. If an interruption occurs,
the RX node saves as much as it has RX'd thus far. After the bus is available
again the RX node is responsible for sending the TX node a message composed of
RX\_address+bits\_thus\_far. If the RX node does not send such a message, the
TX node may solicit one; if the RX node has dropped the suspended transaction
for any reason, it simply responds 0~bits.

Nodes that wish to permit multiple simultaneous bulk transfers could expose
several bulk addresses. If an A$\rightarrow$B transfer is interrupted and C
attempts a C$\rightarrow$B transfer, B would indicate busy by interrupting
immediately after the bulk address was received with a {\tt 01} RX~Error
Interrupt sequence.

\paragraph{Full addresses not unique?} Probably fair to make the assertion
that only one functional unit per node can accept bulk transfers. Repurpose
the functional unit id field to contain the short prefix?
