\svnInfo $Id$

\section{ToDo}
\label{sec:todo}

\subsection{Future Extensions}
\label{sec:todo-extensions}
There are properties that would be nice to have, but introduce greater
complexity. While they are not in the current design, some methods for
designing them are considered here such that they are still feasible for
future implementation in a backwards-compatible manner.

\subsubsection{Automatically fragment long \proto messages}
Add a configuration register for a maximum message length, maybe one for each
message type, that limits maximum transaction length. Requests to send
messages over this length should be automatically fragmented by the sender.

\subsubsection{Full Address Support in \proto}
\label{sec:todo-full-addr}
Currently there is no way to generate a response that sends a message to a
full address. A full prefix is 24~bits long. The following are some possible
methods to support full addresses:
\begin{enumerate}
  \item {\bf Use new FU\_IDs.} Easiest solution, but uses a lot of the
    remaining FU\_IDs. Could do something creative such as the full prefix
    (24~bits) is sent and then the next 3~or 4~bits map the remaining words
    onto one of the existing FU\_IDs. This wastes at least the 4~bits that
    previously specified the short prefix.
  \item {\bf Full Prefix Register(s).} Store a full prefix in a register.
    Since registers are also 24~bits, they can hold a full prefix exactly.
    This is actually a little unfortunate, since a destination FU\_ID also
    needs to be specified. While the short prefix {\tt 1111} in the current
    message can be used as a sentinel, something needs to indicate which
    register the full prefix should be read from. The FU\_ID field could be
    repurposed to this end, but then there is nothing to specify the FU\_ID of
    the eventual message. This could spill over to two registers, but that is
    an inefficient use of registers. Full prefix registers also become another
    contested, shared resource with this scheme.
  \item {\bf Optional Extra Word.} Use the existing short prefix as a
    sentinel. If set to {\tt 1111}, the (next, last) word will be the full
    prefix. Since the short prefix is the first 4~bits, could also just
    ``replace'' the first word of the message with a full address---this has
    nice symmetry with how short/full prefixes are interpreted on the bus. In
    any of these cases, an extra 8~bits become available for some other
    purpose. The runtime variable length message can be tricky to implement.
    Injecting an extra word into the middle of a message makes it harder for
    simple devices with fixed send buffers to support these messages.
\end{enumerate}

\subsubsection{Reg \#240: Register Message Control}
\label{cmd:conf-reg-ctrl}

\begin{bytefield}{8}
  \bitheader{0-7} \\
  \colorbitbox{lightgreen}{2}{11}
  \colorbitbox{lightergreen}{6}{110000}
\end{bytefield}
~
\begin{bytefield}{24}
  \bitheader{0-23} \\
  \bitbox{1}{\begin{sideways}{\tiny EN}\end{sideways}}
  \colorbitbox{lightgray}{23}{Reserved}
\end{bytefield}
\hfill\textbf{Default: \texttt{0x800000}}
\\

\begin{itemize}
  \item {\tt \{R[23]\}: Enable (EN).}
    \subitem Controls whether register writes to this device are enabled. If
    {\tt EN} is {\tt 0}, attempts to write registers~\#0--247 are silently
    ignored.
    \subitem {\bf Note:} Registers~\#248--255 ({\tt b'11111XXX}) can always be
    written.
    \subitem {\bf Acknowledement/Interjection:} Since a single register write
    transaction could write both enabled and disabled registers, a node should
    not interject when a disabled register is written. A node should ACK to
    indicate that the transmission was received successfully, even if nothing
    is actually written.
    {\em This behavior is subject to change in future revisions of \proto.}
    \subitem {\color{red} \bf Warning:} If {\tt EN} is disabled, it cannot be
    re-enabled by a write to this register since register writes are disabled.
    Access can be recovered using \nameref{cmd:conf-reg-reset}, however this
    will reset all registers.
    \subitem \hl{TODO} What about a local write (e.g. a register write via an
    interrupt from the CPU)? Should probably always allow those. Does that
    make sense as an exception?
\end{itemize}
