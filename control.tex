\svnInfo $Id$

\section{Control and Configuration Messages}
\label{sec:control}

\bus reserves three addresses: a {\em broadcast} address, the {\em control}
address, and an {\em extension} address. A single physical chip acting as
control node will have two addresses. The control chip should respond to
messages addressed to the bus controller at the control address and messages
addressed to the chip itself at the chip's unique address. This separation
permits a standard set of commands for \bus control nodes without restricting
the semantics of the data sent to a control node implementation.

\subsection{Broadcast (Address \texttt{0x00}, \texttt{0xf0000000})}
\label{sec:control-broadcast}
\bus defines address 0 as the \textit{broadcast address}. Broadcast messages
are permitted to be of arbitrary length. Messages longer than 32~bits may be
silently dropped by nodes with small buffers. A node \textbf{must not}
interrupt a broadcast message to indicate buffer overflow. Other interrupts
are permitted for messages greater than 4~bytes in length.

\bus reserves half of the broadcast ``data space'' as well. The MSB of the
broadcast data (the first data bit transmitted) shall identify \bus broadcast
operations. If the MSB is {\tt 0} it indicates an official \bus broadcast
message as specified in this document and subsequent revisions. Broadcast
messages with an MSB of {\tt 1} are implementation defined.

A broadcast message that is not understood \textbf{must} be completely
ignored. During acknowledgement, an ignorant node shall forward.

\subsubsection{Query Devices (Data \texttt{0x00000000})}
The query devices command is a request for all devices to broadcast their full
address on the bus. Every \bus node should prepare a Query~Response when this
message is received.

\medskip
\noindent
\textit{All nodes are required to support this message.}

\subsubsection{Query Response (Data \texttt{0x7xxxxxxx})}
This message is sent in response to a Query~Devices request. As every node
will be transmitting their address, nodes should anticipate losing arbitration
several times before they are able to send their response.

The top four bits of the data field identify the message as a Query~Response.
The remaining 28~bits contain the full address of the responding node.

\subsection{Control (Address \texttt{0x01}, \texttt{0xf0000001})}
\label{sec:control-control}
\bus defines a set of common command messages to configure the bus. The chip
acting as the control node listens on the address {\tt 0x01} for \bus control
messages.

Control messages that seem like good ideas (get? set?):
\begin{itemize}
  \item Maximum message length
  \item Clock speed
  \item $t_{long}$ value?
  \item \ldots?
\end{itemize}

\subsection{Extension (Address \texttt{0xffffffff})}
This address is reserved for future extensions to \bus.

