\svnInfo $Id$

\section{Control and Configuration Messages}
\label{sec:control}

\bus reserves three addresses: a {\em broadcast} address, the {\em control}
address, and an {\em extension} address. A single physical chip acting as
control node will have (at least) two addresses. The control chip should
respond to messages addressed to the bus controller at the control address and
messages addressed to the chip itself at the chip's unique address. This
separation permits a standard set of commands for \bus control nodes without
imposing limits on the messages sent to the chip that happens to house the
\bus controller.

\subsection{Broadcast (Address \texttt{0x0X}, \texttt{0xf000000X})}
\label{sec:control-broadcast}
\bus defines prefix 0 as the \textit{broadcast prefix}. Broadcast messages
are permitted to be of arbitrary length. Messages longer than 32~bits may be
silently dropped by nodes with small buffers. A node \textbf{must not}
interrupt a broadcast message to indicate buffer overflow. Other interrupts
are permitted for broadcast messages greater than 4~bytes in length.

As broadcast messages by definition target all endpoints, assigning functional
unit ID targets does not make sense. Instead, the functional unit ID field is
used to define broadcast {\em channels}. Broadcast channel selection is used
to differentiate between the different types of broadcast messages. \bus
reserves half of these channels and leaves the rest as implementation defined.
%
The MSB of the broadcast channel identifier (address bit~3) shall identify
\bus broadcast operations. If the MSB is {\tt 0} it indicates an official \bus
broadcast message as specified in this document and subsequent revisions.
Broadcast messages with a channel MSB of {\tt 1} are implementation defined.

A broadcast message that is not understood \textbf{must} be completely
ignored. During acknowledgement, an ignorant node shall forward.

\hl{TODO: Semantics of broadcast messages in the face of power-gated nodes.}

\subsubsection{Broadcast Messages and Power-Gating}
As power-gating of nodes is a \bus design consideration, \bus also defines the
responsibilities of power-gated nodes for broadcast messages. The {\tt
Bus~Controller} {\bf must} wake for every message. Wakeup shall occur during
the arbitration phase of a message. For non-broadcast messages, the {\tt
Bus~Controller} decides whether it will wake the node once it has received the
entire address. To simplify {\tt Bus~Controller} design, the same constraint
is imposed on broadcast messages: the {\tt Bus~Controller} should be able to
decide whether it will wake the node after receiving the complete address.
The implication is that the rules for waking nodes in response to broadcast
messages are grouped by broadcast channel. This requirement {\bf must} hold
true for all \bus broadcast channels and should hold true for any
implementation-defined channels.

Some messages (e.g. \nameref{cmd:query-devices}) require a response, but may
not require waking the node. The message semantics take priority over the
channel power rules. That is, if a message must be responded to and the {\tt
Bus~Controller} logic is too simple to respond, the {\tt Bus~Controller} {\bf
must} wake the node to respond.

The \bus broadcast channel power rules are summarized here:
\begin{itemize}
  \item \nameref{sec:channel-0}
  \begin{itemize}
    \item Nodes responsible for performing enumeration {\bf must} be woken.
    \item All other nodes may be woken.
  \end{itemize}
  \item \nameref{sec:channel-1}
  \begin{itemize}
    \item \dots
  \end{itemize}
  \item \nameref{sec:channel-2-7}
  \begin{itemize}
    \item Nodes capable of receiving arbitrary messages (e.g. CPU) may be
             woken.
    \item Fixed-function nodes (no software) {\bf must not} be woken.
  \end{itemize}
\end{itemize}

\subsubsection{Broadcast Channels and Messages}
This section breaks down all of the defined broadcast channels and messages.
All undefined channels are reserved and shall not be used. A node receiving a
broadcast message for a reserved channel shall ignore the message. It {\bf
must not} acknowledge the message, it {\bf must} forward during the
acknowledgement cycle. It should not wake a power-gated node to alert of the
broadcast message.

All examples are shown with short addresses for space. There is no distinction
between the use of the short or full broadcast address.

\paragraph{Broadcast Channel 0: Node Discovery and Enumeration}
\label{sec:channel-0}

\subparagraph{Query Devices (Data \texttt{0x00000000})}
\label{cmd:query-devices}
~

\begin{figure}[h]
  \begin{subfigure}{.2\linewidth}
    \centering
    \begin{bytefield}{8}
      \bitheader{0-7} \\
      \bitbox{4}{0000}
      \bitbox{4}{0000}
    \end{bytefield}
    \caption{Address}
  \end{subfigure}
%
  \begin{subfigure}{.8\linewidth}
    \centering
    \begin{bytefield}[bitwidth=1.25em]{32}
      \bitheader{0-31} \\
      \bitbox{4}{0000}
      \bitbox{24}{0000 0000 0000 0000 0000 0000 0001}
      \bitbox{4}{0000}
    \end{bytefield}
    \caption{Data}
  \end{subfigure}
\end{figure}

The query devices command is a request for all devices to broadcast their
static full prefix and currently assigned short prefix on the bus. Every \bus
node must prepare a \nameref{cmd:query-response} when this message is
received.

\medskip
\noindent
\textit{All nodes are required to support this message and respond.}

\subparagraph{Enumerate Node (Data \texttt{0x0000001x})}
\label{cmd:enumerate-node}
~

\begin{figure}[h]
  \begin{subfigure}{.2\linewidth}
    \centering
    \begin{bytefield}{8}
      \bitheader{0-7} \\
      \bitbox{4}{0000}
      \bitbox{4}{0000}
    \end{bytefield}
    \caption{Address}
  \end{subfigure}
%
  \begin{subfigure}{.8\linewidth}
    \centering
    \begin{bytefield}[bitwidth=1.25em]{32}
      \bitheader{0-31} \\
      \bitbox{4}{0000}
      \bitbox{24}{0000 0000 0000 0000 0000 0000 0001}
      \bitbox{4}{Short Prefix}
    \end{bytefield}
    \caption{Data}
  \end{subfigure}
\end{figure}

This message assigns a short prefix to a device. All nodes that receive this
message and do not have an assigned short prefix {\bf must} attempt to reply
with a \nameref{cmd:query-response}. Nodes shall perform exactly one attempt
to reply to this message. The node that wins arbitration shall be assigned the
short prefix from this message. Nodes that lose arbitration shall remain
unchanged.

Nodes that have an assigned short prefix shall ignore this message.

\medskip
\noindent
\textit{All nodes are required to support this message and respond if
appropriate.}

\subparagraph{Invalidate Prefix (Data \texttt{0x0000002x})}
\label{cmd:invalidate-prefix}
~

\begin{figure}[h]
  \begin{subfigure}{.2\linewidth}
    \centering
    \begin{bytefield}{8}
      \bitheader{0-7} \\
      \bitbox{4}{0000}
      \bitbox{4}{0000}
    \end{bytefield}
    \caption{Address}
  \end{subfigure}
%
  \begin{subfigure}{.8\linewidth}
    \centering
    \begin{bytefield}[bitwidth=1.25em]{32}
      \bitheader{0-31} \\
      \bitbox{28}{0000 0000 0000 0000 0000 0000 0002}
      \bitbox{4}{Short Prefix}
    \end{bytefield}
    \caption{Data}
  \end{subfigure}
\end{figure}

This message clears the assignment of a short prefix. The bottom 4~bits
specifiy the node whose prefix shall be reset. A node shall reset its prefix
to \nameref{sec:spec-unassigned-short-prefix}. If the prefix to clear is set
to \nameref{sec:spec-unassigned-short-prefix}, then all nodes shall reset
their prefixes.

\subparagraph{Query/Enumerate Response (Data \texttt{0x1xxxxxxx})}
\label{cmd:query-response}
~

\begin{figure}[h]
  \begin{subfigure}{.2\linewidth}
    \centering
    \begin{bytefield}{8}
      \bitheader{0-7} \\
      \bitbox{4}{0000}
      \bitbox{4}{0000}
    \end{bytefield}
    \caption{Address}
  \end{subfigure}
%
  \begin{subfigure}{.8\linewidth}
    \centering
    \begin{bytefield}[bitwidth=1.25em]{32}
      \bitheader{0-31} \\
      \bitbox{4}{0001}
      \colorbitbox{lightred}{4}{0000}
      \bitbox{20}{Full Prefix}
      \bitbox{4}{Short Prefix}
    \end{bytefield}
    \caption{Data}
  \end{subfigure}
\end{figure}

This message is sent in response to a \nameref{cmd:query-devices} request. As every node
will be transmitting their address, nodes should anticipate losing arbitration
several times before they are able to send their response.

The top four bits of the data field identify the message as a Query~Response.
The next four bits are reserved. The following 20~bits contain the full prefix
of the node. The final 4~bits are the currently assigned short prefix. Nodes
that have not been enumerated should report a short prefix of {\tt 0b1111}.

This message must be sent in response to \nameref{cmd:query-devices} or
\nameref{cmd:enumerate-device}. The node that
issued Query~Devices \textbf{must} acknowledge the response message, other
nodes may ignore or acknowledge a Query~Response. A NAK'd Query~Response is
\textbf{not} retried. An Query~Response that is Interrupted be \textbf{must}
retried until a ACK or NAK is received.

\medskip
\noindent
\textit{All nodes are required to support this message.}

\paragraph{Broadcast Channel 1: Power}
\label{sec:channel-1}

\paragraph{Broadcast Channels 2-7: Reserved}
\label{sec:channel-2-7}

\subsection{Control (Address \texttt{0x01}, \texttt{0xf0000001})}
\label{sec:control-control}
\bus defines a set of common command messages to configure the bus. The chip
acting as the control node listens on the address {\tt 0x01} for \bus control
messages.

Control messages that seem like good ideas (get? set?):
\begin{itemize}
  \item Maximum message length
  \item Clock speed
  \item $t_{long}$ value?
  \item \ldots?
\end{itemize}

\subsection{Extension (Address \texttt{0xffffffff})}
This address is reserved for future extensions to \bus.

