\newcommand{\figTimingArbitration}{%
  \centering
  \footnotesize
  \begin{tikztimingtable}[timing/wscale=6.0,timing/slope=.3]
    %          IrrrrrrRLLL A dddddddddD L ttttttttT 0
    CLK In   & H      HHHC C          C C         C C\\
    CLK Out  & H      HHHC C          C C         C C\\
    Data In  & H      HHHH H .2H.2L.6H  H .4H.6L    L\\
    Data Out & H.5H.5L LLL L L          L .4H.6L    L\\
             & \\
    \\
    CLK In   & H      HHHC C           CC         C C\\
    CLK Out  & H      HHHC C           CC         C C\\
    Data In  & H.7H.3L LLL L L          L .4L.2H.4L L\\
    Data Out & H.7H.3L LLL L L          L .4L.2H.4L L\\
             & \\
    \\
    CLK In   & H      HHHC C           CC         C C\\
    CLK Out  & H      HHHC C           CC         C C\\
    Data In  & H.8H.2L LLL L L          L .6L.2H.2L L\\
    Data Out & H      LLLL L H          H L         L\\
             & \\
    \\
    CLK In   & H      HHHC C           CC         C C\\
    CLK Out  & H      HHHC C           CC         C C\\
    Data In  & H.2H.8L LLL L .2L.8 H    H .2H.8L    L\\
    Data Out & H      HHHH H .2L.8 H    H .2H.8L    L\\
    Int Clk  & X      XXCC C           CC         C C\\
             & \\
  \extracode
    \begin{pgfonlayer}{background}
      \begin{scope}[semithick,dashed]
        \vertlines[color=gray]{6,30,36,...,\twidth}
        \vertlines[color=red]{36,48,...,\twidth}
      \end{scope}
    \end{pgfonlayer}

    \begin{scope}
      [font=\bf\sffamily,shift={(-5.5em,-1.5)},anchor=east,color=blue]
      \node [rotate=45] at (  0,   0) {1};
      \node [rotate=45] at (  0, -12) {2};
      \node [rotate=45] at (  0, -24) {3};
      \node [rotate=45] at (  0, -36) {Ctl};
    \end{scope}

    \begin{scope}
      [font=\sc\scriptsize,shift={(-1,5.5)},anchor=north,align=center]
      \node [rotate=45] at (30, 0) {Arbi-\\tration};
      \node [rotate=45] at (36, 0) {Prio\\Drive};
      \node [rotate=45] at (42, 0) {Prio\\Latch};
      \node [rotate=45] at (48, 0) {Drive\\Bit~0};
      \node [rotate=45] at (54, 0) {Latch\\Bit~0};
      \node [rotate=45] at (60, 0) {Drive\\Bit~1\ldots};
    \end{scope}

    \begin{scope}
      [font=\scriptsize]
      \draw[orange,thick]  (30,  -5) ellipse (1.25 and 2.5);
      \draw[orange,thick]  (30, -29) ellipse (1.25 and 2.5);
      \draw(26.5,-8.5)  node[align=center] {Won\\Arbitration};
      \draw(26.5,-32.5) node[align=center] {Lost\\Arbitration};

      \draw[cyan,thick]    (36,  -5) ellipse (1.25 and 2.5);
      \draw(36, -9) node[align=center,fill=white,fill opacity=.75] {Does Not\\Forward};

      \draw[olive,thick]   (42,  -5) ellipse (1.25 and 2.5);
      \draw[olive,thick]   (42, -29) ellipse (1.25 and 2.5);
      \draw(45,-8.5)  node[align=center] {Prio Req\\Back Off};
      \draw(45,-32.5) node[align=center] {Won Prio\\Arbitration};
    \end{scope}

    \begin{scope}
      [color=blue]
      \draw
        (7.2,-39) node[] (TL) {}
        (7.2,-50) node[] (BL) {}
        ( 24,-36) node[] (TR) {}
        ( 24,-50) node[] (BR) {};
      \node[right=6 of BL] (tlong) {$t_{long}$};
      \draw[<-] (BL.east) -- (tlong.west);
      \draw[->] (tlong.east) -- (BR.west);
      \draw[dashed] (TL) -- (BL);
      \draw[dashed] (TR) -- (BR);
    \end{scope}

    \draw (6, -46.5) -- +(0,-.2)
      node [below,inner sep=2pt] {\scalebox{.75}{\footnotesize0}};
    \foreach \n [evaluate=\n as \l using int((\n-24)/6)] in {30,36,...,\twidth}
      \draw (\n,-46.5) -- +(0,-.2)
        node [below,inner sep=2pt] {\scalebox{.75}{\footnotesize\l}};
  \end{tikztimingtable}
  \caption{
    \bus Arbitration. To begin a transaction, nodes pull down on data.
    Here we show node~1 and node~3 requesting the bus at nearly the same time
    (node~1 shortly after node~3). Node~1 initially wins arbitration, but
    node~3 uses the priority arbitration cycle to claim the bus. The
    propogation delay of the data line between nodes is exaggurated to show
    the shoot-through nature of \bus.
  }
  \label{fig:arbitration}
} % End \figTimingArbitration


\newcommand{\figTimingInterrupt}{%
\hspace{-1em}
\noindent\makebox[\textwidth]{%
  \footnotesize
  \begin{tikztimingtable}[timing/wscale=3.0,timing/slope=.3]
    %          10011 EE EE IIIIII SS X110 0II
    CLK In   & CCCCC CC CC CHHHHH HC CCCC CCC\\
    CLK Out  & CCCCC CC CC CHHHHH HC CCCC CCC\\
    Data In  & HHLLH HH HH HCCCCC CH HHHL LHH\\
    Data Out & HHLLH HH HH HCCCCC CH HHHL LHH\\
             & {1D{Data 1}}{2D{Data 0}}{2D{Data 1}}
               {2D{Data 1}}{2D{Data 1}}{6D{Interrupt}}{2D{Switch Role}}
               {2D{Control 1}}{2D{ACK=True}}{2D{Idle}}{1D{\ldots}} \\
    \\
    % TX Node
    CLK In   & CCCCC CC CC CHHHHH HC CCCC CCC\\
    CLK Out  & CCCCC CC LL LLLLLL LL CCCC CCC\\
    Data In  & HHLLH HH HH HCCCCC CH HHHL LHH\\
    Data Out & HHLLH HH HH HCCCCC CH HHHL LHH\\
             & {1D{Data 1}}{2D{Data 0}}{2D{Data 1}}
               {4D{Req Interrupt}}{6D{Interrupt}}{2D{Switch Role}}
               {2D{EoM=True}}{2D{Control 0}}{2D{Idle}}{1D{\ldots}} \\
    \\
    CLK In   & CCCCC CC LL LLLLLL LL CCCC CCC\\
    CLK Out  & CCCCC CC LL LLLLLL LL CCCC CCC\\
    Data In  & HHLLH HH HH HCCCCC CH HHHL LHH\\
    Data Out & HHLLH HH HH HCCCCC CH HHHL LHH\\
             & {1D{Data 1}}{2D{Data 0}}{2D{Data 1}}
               {10D{Interrupt}}{2D{Switch Role}}
               {2D{Control 1}}{2D{Control 0}}{2D{Idle}}{1D{\ldots}} \\
    \\
    % CTL Node
    CLK In   & CCCCC CC LLL LLLLL LL CCCC CCC\\
    CLK Out  & CCCCC CC CCC HHHHH HC CCCC CCC\\
    Data In  & HHLLH HH HHH CCCCC CH HHHL LHH\\
    Data Out & HHLLH HH HHH CCCCC CH HHHL LHH\\
    Int Clk  & CCCCC CC CCC CCCCC CC CCCC CCC\\
             & {1D{Data 1}}{2D{Data 0}}{2D{Data 1}}
               {4D{Enter Interrupt}}{6D{Interrupt}}{2D{Switch Role}}
               {2D{Control 1}}{2D{Control 0}}{2D{Idle}}{1D{\ldots}} \\
  \extracode
    \begin{pgfonlayer}{background}
      \begin{scope}[semithick,dashed]
        \vertlines[color=red]{3,9,...,27}
        \vertlines[color=gray]{6,12,...,15}
        \vertlines[color=blue]{45,51,...,\twidth}
        \vertlines[color=OliveGreen]{48,54,...,\twidth}

        \filldraw[yellow,opacity=.25] (15, 2) rectangle (27, -45);
      \end{scope}
    \end{pgfonlayer}

    \draw[red,thick]    (21,  -2.5) ellipse (1.25 and 4.5);
    \draw[red,thick]    (27,  -2.5) ellipse (1.25 and 4.5);
    \draw[blue,thick]   (45,  -2.5) ellipse (1.25 and 4.5);
    \draw[green,thick]  (45, -26.5) ellipse (1.25 and 4.5);
    \draw[cyan,thick]   (24, -36.5) ellipse (1.25 and 2.5);
    \draw[cyan,semithick,dashed] (24,-44.5) to (24,-33.25); % .25 better dash
    \node at (18, -7.5)  {\huge\color{blue} X};
    \node at (24, -7.5)  {\huge\color{blue} X};
    \node at (18, -31.5) {\huge\color{green} $\checkmark$};
    \node at (24, -31.5) {\huge\color{green} $\checkmark$};

    \begin{scope}
      [font=\bf\sffamily,shift={(-5.5em,-1.5)},anchor=east,color=blue]
      \node [rotate=45] at (  0,   0) {1};
      \node [rotate=45] at (  0, -12) {2 (TX)};
      \node [rotate=45] at (  0, -24) {3};
      \node [rotate=45] at (  0, -36) {Ctl};
    \end{scope}

    \foreach \n [evaluate=\n as \l using int((\n-1)/3)] in {3,6,...,\twidth}
      \draw (\n,-46.5) -- +(0,-.2)
        node [below,inner sep=2pt] {\scalebox{.75}{\footnotesize\l}};
  \end{tikztimingtable}
}
  \caption{
    Detail of the Interrupt entry procedure and control bits. In this example,
    Node~2 is the TX node, thus when it elects to enter Interrupt it is
    already forwarding its data lines. Node~2 decides to enter Interrupt at
    time~4. At time~6 Node~2 suppresses the clock edge, requesting Interrupt.
    The control layer detects this at time~7\protect\footnotemark.  At this
    point the control node begins toggling the {\em data} lines to interrupt
    the bus. After three data pulses, all nodes will enter Interrupt. Two
    control bits are sent, after which the bus returns to Idle.
  }
  \label{fig:interrupt}
} % End \figTimingInterrupt
