%\svnInfo $Id$

\section{Addressing Design}
\label{sec:addressing}

One of the \bus design points is to serve as a system interconnect for a
physically constrained system. With that in mind, \bus attempts to optimize
for a system of complex connected components. In particular, \bus expects that
an individual member node may itself be composed of multiple functional units,
which may each be individually addressed.

Supporing multiple, addressable functional units with only one set of exposed
\bus signal pins could also be done by instantiating several copies of a
member node on each die, but this requires a bus interface module for every
functional unit within an integrated chip. Instead, the bottom four bits of
addresses are reserved to identify functional units. Nodes with more than
sixteen functional units will require multiple addresses.

\subsection{Address Types}
\label{sec:addressing-types}

\bus defines the term {\em prefix} to refer to the portion of the address that
specifies which node is being addressed and reserves the term {\em address} to
refer to a complete addres---one that specifies a functional unit within a
node. An address is the composition of a prefix with a functional unit
identifier.

Every \bus node has two prefixes, a {\em full} prefix and a {\em short}
prefix. With a functional unit identifier, these become a full address and a
short address. A short prefix is 4~bits long and a full prefix is 28~bits
long. To distinguish full and short prefixes, the short prefix {\tt 0b1111} is
reserved to identify a full prefix. All long prefixes begin with {\tt 0b1111}
and therefore have only $2^{24}$ unique bits.


\subsection{Full Prefix Assignment}
\label{sec:addressing-full}

\begin{bytefield}[bitwidth=1.5em]{32}
  \bitheader{0-31} \\
  \colorbitbox{lightgreen}{4}{0xF}
  \colorbitbox{lightergreen}{4}{0x0 (RSVD)}
  \bitbox{20}{Full Prefix}
  \bitbox{4}{Func Unit ID}
\end{bytefield}

The purpose of full prefixes is to serve as a unique identifier for a node,
akin to a product identifier. Full prefixes do not distinguish instantiations
of a node, that is, multiple copies of a unique part will all have the same
full prefix. Bits~24-27 of the full address are reserved for other purposes
(\ref{sec:todo-extensions-resume}~\nameref{sec:todo-extensions-resume}). The
remaining range, bits~4-23, are available to be utilized as full prefixes.

The full prefixes ranging {\tt 0x00000-0x0000F} are reserved for legacy M3
devices. \hl{An allocation scheme for new full prefixes is currently
undefined.}

\subsection{Short Prefix Assignment}
\label{sec:addressing-short}
