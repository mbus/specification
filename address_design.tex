\svnInfo $Id$

\section{Addressing Design}
\label{sec:addressing}

One of the \bus design points is to serve as a system interconnect for a
physically constrained system. With that in mind, \bus attempts to optimize
for a system of complex connected components. In particular, \bus expects that
an individual member node may itself be composed of multiple functional units,
which may each be individually addressed.

Supporing multiple, addressable functional units with only one set of exposed
\bus signal pins could be done by instantiating several copies of a
member node on each die, but this requires a bus interface module for every
functional unit within an integrated chip. Instead, the bottom four bits of
addresses are reserved to identify functional units. Nodes with more than
sixteen functional units will require multiple addresses.

\subsection{Address Types}
\label{sec:addressing-types}

\bus defines the term {\em prefix} to refer to the portion of the address that
specifies which node is being addressed and reserves the term {\em address} to
refer to a complete address---one that specifies a functional unit within a
node.

Every \bus node has two prefixes, a {\em full} prefix and a {\em short} prefix.
A short prefix is 4~bits long and a full prefix is 20~bits long.
A short address is the composition of short prefix and a functional
unit identifier. A full address is the composition of a header, a full prefix,
and a functional unit identifier.
To distinguish full and short addresses, the short prefix {\tt 0b1111} is
reserved to identify a full address. The first four bits of the full address
header is thus always {\tt 0b1111}.


\subsection{Full Prefix Assignment}
\label{sec:addressing-full}

\begin{center}
\begin{bytefield}[bitwidth=1.4em]{32}
  \bitheader{0-31} \\
  \colorbitbox{lightgreen}{4}{0xF}
  \colorbitbox{lightergreen}{4}{0x0 (RSVD)}
  \bitbox{20}{Full Prefix}
  \colorbitbox{lightcyan}{4}{Func Unit ID}
\end{bytefield}
\end{center}

The purpose of full prefixes is to serve as a unique identifier for a node,
akin to a product identifier. Full prefixes do not distinguish instantiations
of a node, that is, multiple copies of a unique part will all have the same
full prefix. This implies that multiple nodes in a single \bus instantiation
may have the same full prefix. Bits~24-27 of the full address are reserved for
other purposes
(\ref{sec:todo-extensions-resume}~\nameref{sec:todo-extensions-resume}). The
remaining range, bits~4-23, are available to be utilized as full prefixes.
If a node has more than 16 functional units, it may have multiple full
prefixes. These full prefixes should be sequential.
\begin{itemize}
\item The full prefix {\tt 0x00000} is reserved as the broadcast prefix.
\item The full prefix {\tt 0x00001} is reserved as the control node prefix.
\end{itemize}

The full prefixes ranging {\tt 0x00000-0x0000F} are reserved for legacy M3
devices. \hl{An allocation scheme for new full prefixes is currently
undefined.}

\subsection{Short Prefix Assignment}
\label{sec:addressing-short}

\begin{center}
\begin{bytefield}[bitwidth=1.5em]{8}
  \bitheader{0-7} \\
  \bitbox{4}{Short Prefix}
  \colorbitbox{lightcyan}{4}{Func Unit ID}
\end{bytefield}
\end{center}

The purpose of short prefixes is to uniquly identify nodes in an \bus system.
Multiple nodes in a \bus instantiation MUST NOT have the same short prefix.
\begin{itemize}
\item The short prefix {\tt 0b0000} is reserved as the broadcast prefix.
\item The short prefix {\tt 0b0001} is reserved as the control node prefix.
\item The short prefix {\tt 0b1111} is reserved to distinguish full addresses.
\end{itemize}
This leaves a remainder of 13 unique short prefixes. These short prefixes map
to actual nodes instantiated in a \bus system. If there are multiple copies of
the same node type (e.g. several external memory chips), each instance is
given a unique short prefix. If a node has greater than 16 functional units
it will be assigned multiple short prefixes, each mapping to one of the node's
full prefixes.

Short prefixes are assigned dynamically.

