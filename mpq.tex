\newcommand{\mpqrecord}{{\bf \color{blue} Record}\xspace}

\section{\proto: Point-to-Point Message Protocol}
\bus also defines a common point-to-point messaging protocol: \proto.
%For those chips kind enough to mind their P's and Q's on MBus
Nodes are not required to support \proto to be \bus compliant, however it is
strongly encouraged.

\proto defines two classes of data: register data and memory data. \proto
registers have 8~bit addresses and are 24~bits wide. \proto memory has 32~bit
addresses and stores data that is 32~bits wide. The register address space and
memory address space are separate constructs, that is register address {\tt
0x00} need not map to memory address {\tt 0x00000000}, although aliasing is
permitted.

\subsection{\proto Registers (Reg \#192--255)}
\label{cmd:mmap}
\begin{minipage}{\linewidth}
  \begin{varwidth}[b]{.55\linewidth}

\proto reserves the top 64~registers for control and configuration.

\medskip
This space configures various \proto capabilities. Much of the space is
currently reserved for future expansion.

\medskip
Attempts to write configuration for unsupported features (e.g.\ configuration
of a memory streaming channel on a device with no memory) is undefined.
Attempts to read unsupported features {\bf must} NAK or return all {\tt 0}.

\medskip
This section documents each of the registers. The top two bits of each \proto
register address are \hlc[lightgreen]{\texttt{11}} followed by six bits that
identify the \hlc[lightergreen]{type}. All \hlc[lightgray]{reserved} bits
should be treated as RZWI.

\medskip
The \proto controller stores state about its current operation in the status
registers, \#240--247. Each \proto operation describes what the \proto
controller will \mpqrecord and when it will trigger an interrupt.
\nameref{cmd:conf-proto-int} controls what events generate interrupts.
  \end{varwidth}
~
  \begin{varwidth}[t]{.45\linewidth}
\centering
\begin{bytefield}[bitwidth=0.6em]{24}
  \bitbox[]{8}{~}\bitbox[]{24}{\bf \proto Register Map}\\
  \begin{rightwordgroup}{\begin{sideways}\small Tie to {\tt 0}\end{sideways}}
    \memsectionhuge{\#192}{\#215}{16}{Reserved}
  \end{rightwordgroup}\\
  \begin{rightwordgroup}{\begin{sideways}\small Retentive\end{sideways}}
    \bitbox[]{8}{\tt\#216}\wordbox{1}{\hyperref[cmd:conf-mem-ctrl]{Bulk Memory Message Control}}\\
    \bitbox[]{8}{\tt\#217}\wordbox{1}{Reserved}\\
    \bitbox[]{8}{\tt\#218}\wordbox{1}{Reserved}\\
    \bitbox[]{8}{\tt\#219}\wordbox{1}{Reserved}\\
    \bitbox[]{8}{\tt\#220}\wordbox{1}{\hyperref[cmd:conf-proto-int]{\proto Record and Interrupt Control}}\\
    \bitbox[]{8}{\tt\#221}\wordbox{1}{Reserved}\\
    \bitbox[]{8}{\tt\#222}\wordbox{1}{Reserved}\\
    \bitbox[]{8}{\tt\#223}\wordbox{1}{Reserved}\\
    \memsection{\#224}{\#239}{8}{\hyperref[cmd:conf-memory-stream]{Memory Stream Configuration Registers}}\\
    \memsection{\#240}{\#247}{4}{Status Registers\\\emph{Read Only}\\(\hyperref[cmd:status-register-summary]{summary})}
  \end{rightwordgroup}\\
  \begin{rightwordgroup}{\begin{sideways}\small Non-Retentive\end{sideways}}
    \memsection{\#248}{\#255}{4}{Action Registers}
  \end{rightwordgroup}
\end{bytefield}
  \end{varwidth}
\end{minipage}

\subsubsection{Reserved Registers}
All registers not specified here are reserved.
Writes to reserved registers should be ignored. Reads from reserved registers
should return all {\tt 0}.

\newpage

\subsubsection{Reg \#216: Bulk Memory Message Control}
\label{cmd:conf-mem-ctrl}

\begin{bytefield}{8}
  \bitheader{0-7} \\
  \colorbitbox{lightgreen}{2}{11}
  \colorbitbox{lightergreen}{6}{110010}
\end{bytefield}
~
\begin{bytefield}{24}
  \bitheader{0-23} \\
  \bitbox{1}{\begin{sideways}{\tiny EN}\end{sideways}}
  \bitbox{1}{\begin{sideways}{\tiny CACT}\end{sideways}}
  \colorbitbox{lightgray}{1}{\begin{sideways}{\tiny RSV}\end{sideways}}
  \colorbitbox{lightgray}{1}{\begin{sideways}{\tiny RSV}\end{sideways}}
  \colorbitbox{lightgray}{1}{\begin{sideways}{\tiny RSV}\end{sideways}}
  \colorbitbox{lightgray}{1}{\begin{sideways}{\tiny RSV}\end{sideways}}
  \colorbitbox{lightgray}{1}{\begin{sideways}{\tiny RSV}\end{sideways}}
  \colorbitbox{lightgray}{1}{\begin{sideways}{\tiny RSV}\end{sideways}}
  \bitbox{16}{Length Limit-1}
\end{bytefield}
\hfill\textbf{Def: \texttt{0x800000}}, \texttt{0x000000} if no mem
\\

The register controls the response of this chip upon the receipt of a
\nameref{cmd:mem-bulk-write} command.

\begin{itemize}
  \item {\tt \{R[23]\}: Enable (EN).}
    \begin{itemize}
      \item Controls whether bulk memory transactions written to this device
        are enabled. If {\tt EN} is {\tt 0}, a node must not modify the
        contents of memory in response to a bulk memory transaction.
      \item {\bf Acknowledement/Interjection:} The behavior of a bulk write
        when {\tt EN} is {\tt 0} is undefined. Receivers are encouraged to
        interject and indicate receiver error, however they may exhibit any
        behavior, including ACK'ing the transaction and silently ignoring it.
    \end{itemize}
  \item {\tt \{R[22]\}: Control Active (CACT).}
    \begin{itemize}
      \item If this bit is high, this register's length field acts as a limit
        for the maximum permitted bulk message length. A bulk message is
        allowed to write until this message limit is reached. If more data
        comes, the message {\bf must} be NAK'd. The receiver should interject
        with receiver error as soon as it knows the length limit has been
        exceeded.
    \end{itemize}
  \item {\tt \{R[15:0]\}: Length Limit.}
    \begin{itemize}
      \item The maximum permitted length in words of a memory bulk write
        message to any address.
    \end{itemize}
\end{itemize}

\subsubsection{Reg \#220: \proto Record and Interrupt Control}
\label{cmd:conf-proto-int}

\begin{bytefield}{8}
  \bitheader{0-7} \\
  \colorbitbox{lightgreen}{2}{11}
  \colorbitbox{lightergreen}{6}{011111}
\end{bytefield}
~
\begin{bytefield}{24}
  \bitheader{0-23} \\
  \colorbitbox{lightgray}{1}{\begin{sideways}{\tiny RSV}\end{sideways}}
  \colorbitbox{lightgray}{1}{\begin{sideways}{\tiny RSV}\end{sideways}}
  \colorbitbox{lightgray}{1}{\begin{sideways}{\tiny RSV}\end{sideways}}
  \colorbitbox{lightgray}{1}{\begin{sideways}{\tiny RSV}\end{sideways}}
  \colorbitbox{lightgray}{1}{\begin{sideways}{\tiny RSV}\end{sideways}}
  \colorbitbox{lightgray}{1}{\begin{sideways}{\tiny RSV}\end{sideways}}
  \colorbitbox{lightgray}{1}{\begin{sideways}{\tiny RSV}\end{sideways}}
  \colorbitbox{lightgray}{1}{\begin{sideways}{\tiny RSV}\end{sideways}}
  \colorbitbox{lightgray}{1}{\begin{sideways}{\tiny RSV}\end{sideways}}
  \colorbitbox{lightgray}{1}{\begin{sideways}{\tiny RSV}\end{sideways}}
  \colorbitbox{lightgray}{1}{\begin{sideways}{\tiny RSV}\end{sideways}}
  \colorbitbox{lightgray}{1}{\begin{sideways}{\tiny RSV}\end{sideways}}
  \colorbitbox{lightgray}{1}{\begin{sideways}{\tiny RSV}\end{sideways}}
  \colorbitbox{lightgray}{1}{\begin{sideways}{\tiny RSV}\end{sideways}}
  \colorbitbox{lightgray}{1}{\begin{sideways}{\tiny RSV}\end{sideways}}
  \colorbitbox{lightgray}{1}{\begin{sideways}{\tiny RSV}\end{sideways}}
  \bitbox{1}{\begin{sideways}{\tiny INT}\end{sideways}}
  %\bitbox{1}{\begin{sideways}{\tiny REC}\end{sideways}} % MEM STRM RD CH3
  \bitbox{1}{\begin{sideways}{\tiny INT}\end{sideways}}
  %\bitbox{1}{\begin{sideways}{\tiny REC}\end{sideways}} % MEM STRM RD CH2
  \bitbox{1}{\begin{sideways}{\tiny INT}\end{sideways}}
  %\bitbox{1}{\begin{sideways}{\tiny REC}\end{sideways}} % MEM STRM RD CH1
  \bitbox{1}{\begin{sideways}{\tiny INT}\end{sideways}}
  %\bitbox{1}{\begin{sideways}{\tiny REC}\end{sideways}} % MEM STRM RD CH0
  \bitbox{1}{\begin{sideways}{\tiny INT}\end{sideways}}
  %\bitbox{1}{\begin{sideways}{\tiny REC}\end{sideways}} % MEM RD
  \bitbox{1}{\begin{sideways}{\tiny INT}\end{sideways}}
  %\bitbox{1}{\begin{sideways}{\tiny REC}\end{sideways}} % MEM BULK WR
  \bitbox{1}{\begin{sideways}{\tiny INT}\end{sideways}}
  %\bitbox{1}{\begin{sideways}{\tiny REC}\end{sideways}} % REG RD
  \bitbox{1}{\begin{sideways}{\tiny INT}\end{sideways}}
  %\bitbox{1}{\begin{sideways}{\tiny REC}\end{sideways}} % REG WR
\end{bytefield}
%\hfill\textbf{Default: \texttt{0xXXXXXX}}
\hfill\textbf{Default: \texttt{0x0000XX}}
\\

%Each \proto FU\_ID command has an associated \mpqrecord action. Every time a
%node completes a command, it checks the associated command {\tt REC} bit. If
%high, the \mpqrecord action is performed. Next the node checks the command's
%{\tt INT} bit. If high, the layer owner is interrupted.
%
%Bits~0 and~1 control FU\_ID~0, bits~2 and~3 control FU\_ID~1, and so on.
%
%\paragraph{Default Value:} The default value of this register is
%implementation-dependent. If not otherwise specified by an implementation
%document, the default action is to both record and interrupt for any
%implemented subsystem. As example, a node with support for registers, bulk
%memory, and streaming memory channels~0 and~1 would have a default value of
%{\tt 0x03ffff}.

Each time a node completes a \proto command, it checks the associated command
{\tt INT} bit. If high, the layer owner is interrupted.

\paragraph{Default Value:} The default value is implementation dependent.
It current defaults to all interrupts off ({\tt 0x000000}).

\paragraph{Note:} See \nameref{sec:todo-fine-grain-dma} for possible future
security considerations.

\hl{TODO: What are the ways in which a command that comes in the interrupt
  port from the layer should differ from a command that comes from the bus?
Separate interrupt control?}

\newpage

\subsubsection{Reg \#224--239: Memory Stream Configuration}
\label{cmd:conf-memory-stream}

\begin{bytefield}{8}
  \bitheader{0-7} \\
  \colorbitbox{lightgreen}{2}{11}
  \colorbitbox{lightergreen}{2}{10}
  \colorbitbox{lightcyan}{2}{XX}
  \colorbitbox{lightergreen}{2}{00}
\end{bytefield}
~
\begin{bytefield}{24}
  \bitheader{0-23} \\
  \bitbox{8}{\scriptsize Alert Address}
  \bitbox{14}{Write Buffer [15:2]}
  \colorbitbox{lightgray}{1}{\begin{sideways}{\tiny RSV}\end{sideways}}
  \colorbitbox{lightgray}{1}{\begin{sideways}{\tiny RSV}\end{sideways}}
\end{bytefield}
\hfill\textbf{Default: \texttt{0x000000}}
\\

\noindent
\begin{bytefield}{8}
  \bitheader{0-7} \\
  \colorbitbox{lightgreen}{2}{11}
  \colorbitbox{lightergreen}{2}{10}
  \colorbitbox{lightcyan}{2}{XX}
  \colorbitbox{lightergreen}{2}{01}
\end{bytefield}
~
\begin{bytefield}{24}
  \bitheader{0-23} \\
  \bitbox{8}{\scriptsize Alert Register to Write}
  \bitbox{16}{Write Buffer [31:16]}
\end{bytefield}
\hfill\textbf{Default: \texttt{0x000000}}
\\

\noindent
\begin{bytefield}{8}
  \bitheader{0-7} \\
  \colorbitbox{lightgreen}{2}{11}
  \colorbitbox{lightergreen}{2}{10}
  \colorbitbox{lightcyan}{2}{XX}
  \colorbitbox{lightergreen}{2}{10}
\end{bytefield}
~
\begin{bytefield}{24}
  \bitheader{0-23} \\
  \bitbox{1}{\begin{sideways}{\tiny EN}\end{sideways}}
  \bitbox{1}{\begin{sideways}{\tiny WRP}\end{sideways}}
  \bitbox{1}{\begin{sideways}{\tiny DBLB}\end{sideways}}
  \colorbitbox{lightgray}{1}{\begin{sideways}{\tiny RSV}\end{sideways}}
  \colorbitbox{lightgray}{1}{\begin{sideways}{\tiny RSV}\end{sideways}}
  \colorbitbox{lightgray}{1}{\begin{sideways}{\tiny RSV}\end{sideways}}
  \colorbitbox{lightgray}{1}{\begin{sideways}{\tiny RSV}\end{sideways}}
  \colorbitbox{lightgray}{1}{\begin{sideways}{\tiny RSV}\end{sideways}}
  \bitbox{16}{Buffer Length-1}
\end{bytefield}
\hfill\textbf{Default: \texttt{0x000000}}
\\

\noindent
\begin{bytefield}{8}
  \bitheader{0-7} \\
  \colorbitbox{lightgreen}{2}{11}
  \colorbitbox{lightergreen}{2}{10}
  \colorbitbox{lightcyan}{2}{XX}
  \colorbitbox{lightergreen}{2}{11}
\end{bytefield}
~
\begin{bytefield}{24}
  \bitheader{0-23} \\
  \colorbitbox{lightgray}{8}{\scriptsize Reserved}
  \bitbox{16}{Buffer Offset}
\end{bytefield}
\hfill\textbf{Default: \texttt{0x000000}}
\\

In \proto each node has up to four independent, identical memory streaming
\hlc[lightcyan]{channels}. Each channel has two configuration registers. The
two registers work together to configure each channel.

\begin{itemize}
  \item {\tt \{R01[15:0], R00[15:2], 2'b00\}: Write Buffer.}
    \begin{itemize}
      \item Pointer to the beginning of a buffer in memory.
    \end{itemize}
  \item {\tt \{R00[23:16]\}: Alert Address.}
    \begin{itemize}
      \item Defines where alerts are sent. If the alert prefix (bits 23:20)
        are set to the full-prefix indicator {\tt 1111}, the alert is
        suppressed.
      \item Alerts are sent whenever the end of the buffer is reached.  If
        {\tt DBLB} is active, an alert is also sent when the halfway point of
        the buffer is reached.
      \item When a memory stream alert occurs, a node sends the following
        message:

        \medskip
        \begin{bytefield}{8}
          \bitheader{0-7} \\
          \bitbox{8}{Alert Address} \\
          \bitbox[t]{8}{\hspace{-.5em}$\underbrace{\hspace{8.5em}}_{\text{\small \bus Address}}$}
        \end{bytefield}
        ~
        \begin{bytefield}{32}
          \bitheader{0-31} \\
          \bitbox{8}{\scriptsize Alert Register To Write}
          \bitbox{8}{\scriptsize Alerting Stream Channel}
          \bitbox{1}{\begin{sideways}{\tiny EN}\end{sideways}}
          \bitbox{1}{\begin{sideways}{\tiny WRP}\end{sideways}}
          \bitbox{1}{\begin{sideways}{\tiny DBLB}\end{sideways}}
          \bitbox{1}{\begin{sideways}{\tiny OVFL}\end{sideways}}
          \colorbitbox{lightgray}{12}{Reserved} \\
          \bitbox[t]{32}{$\underbrace{\hspace{33em}}_{\text{\small \bus Data}}$}
        \end{bytefield}
      \begin{itemize}
        \item The \bus Address is set to the {\tt Alert Address} specified by
          bits~23:16 in {\tt R1110XX00}.
        \item The top 8~bits of data are set to the {\tt Alert Register to
          Write} specified by bits~23:16 in {\tt R1110XX01}.
        \item The next 8~bits are the {\tt Alerting Stream Channel}, made up
          of this node's short prefix, followed by {\tt 01}, followed by the
          channel that generated the alert---this should be the same address
          used to write to this stream channel.
        \item The {\tt EN} bit reports the current state of the {\tt EN} bit
          of the alerting channel when the alert was sent.
        \item The {\tt WRP} bit is set if the write that generated this alert
          reached the end of the stream buffer.
        \item The {\tt DBLB} bit is set if the write that generated this alert
          reached the halfway point of the stream buffer and double-buffering
          is active for this stream.
        \item The {\tt OVFL} bit is set if the write that generated this alert
          reached the end of the stream buffer and there was already a pending
          alert with the {\tt WRP} bit set or if the write that generated this
          alert reached the halfway point of the stream buffer and double
          buffering is active for this stream and there was already a pending
          alert with the {\tt DBLB} bit set.
      \end{itemize}
    \item \hl{TODO: What happens if a node wishes to alert itself? e.g. a CPU
        that puts itself to sleep but wants to be interrupted when 1,000
      samples have been written to memory?}
    \end{itemize}
  \item {\tt \{R10[15:0]\}: Buffer Length-1.}
    \begin{itemize}
      \item Defines the length of the buffer.
    \end{itemize}
  \item {\tt \{R10[23]\}: Enable (EN).}
    \begin{itemize}
      \item Controls whether this channel is enabled. If {\tt EN} is {\tt 0},
        a node must not modify memory in response to a memory stream message.%
        \footnote{
          Implementation Tip: Any undefined \proto registers are defined to be
          RZWI, that is a read from an undefined register will read as all 0's
          (and a write ignored). Upon a read, this will return {\tt 0} for
          {\tt EN} as required.  Nodes that do not implement a memory stream
          channel do not require any logic to handle any memory stream
          messages, simply leave the register undefined.
        }
      \item {\bf Acknowledement/Interjection:} The behavior of a stream write
        when {\tt EN} is {\tt 0} is undefined. Receivers are encouraged to
        interject and indicate receiver error, however they may exhibit any
        behavior, including ACK'ing the transaction and silently ignoring it.
    \end{itemize}
  \item {\tt \{R10[22]\}: Wrap (WRP).}
    \begin{itemize}
      \item Defines node behavior when the end of the buffer is reached. If
        {\tt WRP} is high, the {\tt Write Address Counter} should reset to its
        original value. If {\tt WRP} is low, the {\tt Write Address Counter}
        value should be unchanged (it should thus be one past the end of the
        valid buffer) and {\tt EN} should be set to {\tt 0}.
    \end{itemize}
  \item {\tt \{R10[21]\}: Double Buffer (DBLB).}
    \begin{itemize}
      \item Controls double-buffering mode. If double-buffering is active, the
        node should generate an alert halfway through the buffer in addition
        to at the end of the buffer. This mode is most useful when combined
        with {\tt WRP}.
    \end{itemize}
  \item \hl{TODO: Document Buffer Offset}
\end{itemize}


\subsubsection{Reg \#255: Action Register}
\label{cmd:conf-reg-reset}

\begin{bytefield}{8}
  \bitheader{0-7} \\
  \colorbitbox{lightgreen}{2}{11}
  \colorbitbox{lightergreen}{6}{111111}
\end{bytefield}
~
\begin{bytefield}{24}
  \bitheader{0-23} \\
  \bitbox{1}{\begin{sideways}{\tiny RST}\end{sideways}}
  \colorbitbox{lightgray}{3}{RSV}
  \bitbox{1}{\begin{sideways}{\tiny RSTR}\end{sideways}}
  \bitbox{1}{\begin{sideways}{\tiny RSTB}\end{sideways}}
  \bitbox{1}{\begin{sideways}{\tiny RSTS}\end{sideways}}
  \colorbitbox{lightgray}{8}{Reserved}
  \bitbox{1}{\begin{sideways}{\tiny INTO}\end{sideways}}
  \colorbitbox{lightgray}{8}{Reserved}
\end{bytefield}
\hfill\textbf{$<$No Storage$>$}
\\

This register requests that an action be performed. It is an error to request
more than one action in a single request. Actions are processed from MSB to
LSB, that is, if more than one action is requested {\em only} the highest
priority action is actually taken.

A read from this register shall always return all {\tt 0}.

A write of all {\tt 0} to this register shall perform no actions.

If any bit written to this register is non-zero, writing this register {\bf
must} be the last operation in the transaction. The behavoir of anything after
a register action request in the same transaction is undefined.

\begin{itemize}
  \item \texttt{\{R[23]\}: Reset (RST).}
    \begin{itemize}
      \item Reset the entire node. The exact result of a reset is
        implementation-defined, however this request should be the most
        aggresive form of reset available.
    \end{itemize}
  \item \texttt{\{R[19]\}: Reset \proto Registers (RSTR).}
    \begin{itemize}
      \item Reset \proto configuration registers that affect register protocol
        behavior to their default value.
      \item Resets \#223--247.
    \end{itemize}
  \item \texttt{\{R[18]\}: Reset Bulk Registers (RSTB).}
    \begin{itemize}
      \item Reset \proto configuration registers that affect memory bulk
        transfers to their default value.
      \item Resets \#242.
    \end{itemize}
  \item \texttt{\{R[17]\}: Reset Stream Registers (RSTS).}
    \begin{itemize}
      \item Reset \proto configuration registers that affect memory stream
        transfers to their default value.
      \item Resets \#224--239.
    \end{itemize}
  \item \texttt{\{R[8]\}: Interrupt Owner (INTO).}
    \begin{itemize}
      \item Asserts the Layer Owner Interrupt.
    \end{itemize}
\end{itemize}



\newpage

\subsection{Register Commands}
\label{cmd:reg}
\proto register space is an 8~bit addressable array of 24~bit wide registers.
Any undefined bits are treated as RZWI (Read as Zero, Write Ignored).

\subsubsection{Register Write}
\label{cmd:reg-write}

\begin{bytefield}{9}
  \bitheader{0-7} \\
  \bitbox[rbt]{5}{{\dots}Prefix}
  \bitbox{4}{0000} \\
  \bitbox[t]{9}{\hspace{-.5em}$\underbrace{\hspace{9.5em}}_{\text{\small \bus Address}}$}
\end{bytefield}
~
\begin{bytefield}{32}
  \bitheader{0-31} \\
  \bitbox{8}{Address}
  \bitbox{24}{Data}
  \bitbox[lbt]{8}{Address\dots} \\
  \bitbox[t]{40}{$\underbrace{\hspace{42em}}_{\text{\small \bus Data}}$}
\end{bytefield}

Bits~0-23 of the \bus data field are written to the register addressed by
bits~24-31. The write occurs immediately, as soon as the layer controller
receives the message. Multiple registers may be written in a single \bus
transaction by sending multiple data packets. Each 32~bit chunk of data is
treated as if it were an independent transaction.

\newcommand{\recordRegWrite}{%
\begin{mdframed}[backgroundcolor=lightercyan]
  \begin{minipage}[b]{.35\linewidth}
    \mpqrecord and generate interrupt after each register write.
  \end{minipage}
  \hfill
  {\bf Reg \#240:}
  \begin{bytefield}{24}
    \bitheader{0-23} \\
    \bitbox{4}{\scriptsize 4~MSB of\\Prefix}
    \bitbox{4}{0000}
    \colorbitbox{lightgray}{8}{Reserved}
    \bitbox{8}{\scriptsize Register Address Written}
  \end{bytefield}
\end{mdframed}
}
\recordRegWrite

\paragraph{Overflow:} If more data is received after writing the last register
({\tt 0xff}), the destination address wraps and registers continue to be
written, beginning at address {\tt 0x00}.
{\em Tests: \ref{test:reg-write-overflow}.}
%
\paragraph{Unaligned Access:} The behavior of a message ending on a
non-word boundary is undefined.
%
\paragraph{Interjection Semantics:} Each command is applied immediately when
it is received. A four-command message would have a $4\times32=128$~bit data
payload. If 63~bits are received prior to interjection, only the first command
was applied. If 64~bits are received, the first two commands are applied.
{\em Tests: \ref{test:inj-reg-long}, \ref{test:inj-reg-end}.}

\subsubsection{Register Read}
\label{cmd:register-read}

\begin{bytefield}{9}
  \bitheader{0-7} \\
  \bitbox[rbt]{5}{{\dots}Prefix}
  \bitbox{4}{0001} \\
  \bitbox[t]{9}{\hspace{-.5em}$\underbrace{\hspace{9.5em}}_{\text{\small \bus Address}}$}
\end{bytefield}
~
\begin{bytefield}{32}
  \bitheader{0-31} \\
  \bitbox{8}{Start Address\\to Read From}
  \bitbox{8}{Length-1}
  \bitbox{8}{\bus Address\\to Reply To}
  \bitbox{8}{Address to Write\\on Destination} \\
  \bitbox[t]{32}{$\underbrace{\hspace{33em}}_{\text{\small \bus Data}}$}
\end{bytefield}

Bits~24-31 specify the address of the register to be read. Bits~16-23 may be
used to request that multiple registers are sent. This field is a count of the
number of values to be sent less one, that is a value of~0 requests 1~register
is read and a value of 255 requests that all 256 registers are sent. Bits~8-15
are the \bus address the reply is sent to. Bits~0-7 specify the first address
field of the \nameref{cmd:reg-write} response.

The response message is sent to the \bus address specified in bits~8-15 of the
request and its data field is formatted exactly as the
\nameref{cmd:reg-write} command: 8~bit address + 24~bit data. For reads
of more than one register, the address field in the response is incremented
by~1 for each register.

The response always sends the requested length. If a request for register
\#256 would have been made, the request wraps and begins from register \#0.
The destination register address wraps similarly.

\paragraph{Note:} The \bus address to reply to {\bf must} be copied
exactly. The FU\_ID is not required to be \nameref{cmd:reg-write}. For
example, if the FU\_ID is \nameref{cmd:mem-stream-write}, the effect is
dumping the current register state to memory on the target addresss.
{\em Tests: \ref{test:reg-dump-restore}.}

\newcommand{\recordRegRead}{%
\begin{mdframed}[backgroundcolor=lightercyan]
  \begin{minipage}[b]{.35\linewidth}
    \mpqrecord command information. Generate the whole response message.
    Interrupt after sending complete response.
  \end{minipage}
  \hfill
  \begin{minipage}[b]{.6\linewidth}
    {\bf Reg \#240:}
    \begin{bytefield}{24}
      \bitheader{0-23} \\
      \bitbox{4}{\scriptsize 4~MSB of\\Prefix}
      \bitbox{4}{0001}
      \bitbox{8}{\scriptsize \bus Address Replied To}
      \bitbox{8}{\scriptsize Address Written on Destination}
    \end{bytefield}

    \medskip
    {\bf Reg \#241:}
    \begin{bytefield}{24}
      \bitheader{0-23} \\
      \colorbitbox{lightgray}{8}{Reserved}
      \bitbox{8}{\scriptsize Length-1}
      \bitbox{8}{\scriptsize Start Address Read From}
    \end{bytefield}
  \end{minipage}
\end{mdframed}
}
\recordRegRead

\paragraph{Interjection Semantics:} If the reply is interjected, the
transaction is aborted and is {\bf not} retried.

\subsection{Memory Commands}
\label{cmd:mem}
\proto memory space is a 32~bit addressable array of 32~bit words of memory.
Any undefined accesses are treated as RZWI (Read as Zero, Write Ignored).
\proto provides two types of memory commands: bulk and stream. A bulk memory
transaction is a wholly self-contained event designed for DMA of large blocks
of memory. Memory streams pre-configure the stream information in \proto
registers on the destination node and omit it from subsequent transactions,
relying on the receiver to maintain and increment a destination pointer.
Streams are useful for applications such as continuous sampling, where
multiple, short messages are generated.

\subsubsection{Memory Bulk Write}
\label{cmd:mem-bulk-write}

\begin{bytefield}{9}
  \bitheader{0-7} \\
  \bitbox[rbt]{5}{{\dots}Prefix}
  \bitbox{4}{0010} \\
  \bitbox[t]{9}{\hspace{-.5em}$\underbrace{\hspace{9.5em}}_{\text{\small \bus Address}}$}
\end{bytefield}
~
\begin{bytefield}[bitwidth=.4em]{72}
%  \bitheader{0,31} \\
  \bitbox[]{2}{\tiny ~ \\ ~ \\ 31}
  \bitbox[]{1}{\tiny ~ \\ ~ \\ \tt \bf \dots}
  \bitbox[]{26}{}
  \bitbox[]{1}{\tiny ~ \\ ~ \\ 2}
  \bitbox[]{1}{\tiny ~ \\ ~ \\ 1}
  \bitbox[]{1}{\tiny ~ \\ ~ \\ 0}
  \bitbox[]{2}{\tiny ~ \\ ~ \\ 31}
  \bitbox[]{1}{\tiny ~ \\ ~ \\ \tt \bf \dots}
  \bitbox[]{28}{}
  \bitbox[]{1}{\tiny ~ \\ ~ \\ 0}
  \bitbox[]{2}{\tiny ~ \\ ~ \\ 31}
  \bitbox[]{1}{\tiny ~ \\ ~ \\ \tt \bf \dots}
  \\
  \bitbox{30}{DMA Start Address}
  \colorbitbox{lightgray}{1}{\footnotesize \tt Z}
  \colorbitbox{lightgray}{1}{\footnotesize \tt Z}
  \bitbox{32}{Data}
  \bitbox[ltb]{8}{Data\ldots}
  \\
  \bitbox[t]{72}{$\underbrace{\hspace{33em}}_{\text{\small \bus Data}}$}
\end{bytefield}

The first word received is the address in memory to begin writing to. The
bottom two bits of the address field are reserved and {\bf must} be
transmitted as 0. Subsequent words are treated as data. The first word of data
is written to the specified address. The next word of data is written to
address+4 (the next word in memory) and so on. There is no limit on the length
of this message.

\newcommand{\recordMemBulkWrite}{%
\begin{mdframed}[backgroundcolor=lightercyan]
  \begin{minipage}[b]{.35\linewidth}
    \mpqrecord command information. When the write completes, generate an
    interrupt. If an error occurs part way through, the Reg \#250 should
    indicate the number of words actually written.
    \hl{TODO: Use reserved bit(s) to indicate error?}
  \end{minipage}
  \hfill
  \begin{minipage}[b]{.6\linewidth}
    {\bf Reg \#240:}
    \begin{bytefield}{24}
      \bitheader{0-23} \\
      \bitbox{4}{\scriptsize 4~MSB of\\Prefix}
      \bitbox{4}{0010}
      \bitbox{14}{\scriptsize Start Address Written To [15:2]}
      \colorbitbox{lightgray}{1}{\begin{sideways}{\tiny RSV}\end{sideways}}
      \colorbitbox{lightgray}{1}{\begin{sideways}{\tiny RSV}\end{sideways}}
    \end{bytefield}

    \medskip
    {\bf Reg \#241:}
    \begin{bytefield}{24}
      \bitheader{0-23} \\
      \colorbitbox{lightgray}{8}{Reserved}
      \bitbox{16}{\scriptsize Start Address Written To [31:16]}
    \end{bytefield}

    \medskip
    \textbf{Reg \#242, \#243 not written.}

    \medskip
    {\bf Reg \#244:}
    \begin{bytefield}{24}
      \bitheader{0-23} \\
      \colorbitbox{lightgray}{8}{Reserved}
      \bitbox{16}{\scriptsize Length Written-1}
    \end{bytefield}
  \end{minipage}
\end{mdframed}
}
\recordMemBulkWrite

\medskip
\noindent
{\em Implementation Note:} These registers may be updated while the command is
running, so long as they have the correct value once it completes. They are a
good place to store the pointer and counter while the command is active
instead of instantiating dedicated counter and address registers.

\paragraph{Unaligned Access:} The behavior of a message ending on a
non-word boundary is undefined.

\paragraph{Overflow:} If more data is received after writing the last
address in memory ({\tt 0xfffffffc}), the destination address wraps and data
continues to be written, beginning at address {\tt 0x00000000}.
{\em Tests: \ref{test:mem-bulk-overflow}.}

\paragraph{Interjection Semantics:} Each word of data is written to memory
immediately when it is received. A four-word message would have a
$(1+4)\times32=160$~bit data payload. If 95~bits are received prior to
interjection, only the first word was written to memory. If 96~bits are
received, the first two words were written to memory.
{\em Tests: \ref{test:inj-mem-bulk-long}, \ref{test:inj-mem-bulk-end}.}

\newpage

\subsubsection{Memory Read}
\label{cmd:mem-read}

\begin{bytefield}{9}
  \bitheader{0-7} \\
  \bitbox[rbt]{5}{{\dots}Prefix}
  \bitbox{4}{0011} \\
  \bitbox[t]{9}{\hspace{-.5em}$\underbrace{\hspace{9.5em}}_{\text{\small \bus Address}}$}
\end{bytefield}
~
\begin{bytefield}[bitwidth=.4em]{96}
%  \bitheader{0,31} \\
  \bitbox[]{2}{\tiny ~ \\ ~ \\ 31}
  \bitbox[]{1}{\tiny ~ \\ ~ \\ \tt \bf \dots}
  \bitbox[]{3}{}
  \bitbox[]{2}{\tiny ~ \\ ~ \\ 24}
  \bitbox[]{2}{\tiny ~ \\ ~ \\ 23}
  \bitbox[]{2}{\tiny ~ \\ ~ \\ 20}
  \bitbox[]{2}{\tiny ~ \\ ~ \\ 19}
  \bitbox[]{1}{\tiny ~ \\ ~ \\ \tt \bf \dots}
  \bitbox[]{16}{}
  \bitbox[]{1}{\tiny ~ \\ ~ \\ 0}
  \bitbox[]{2}{\tiny ~ \\ ~ \\ 31}
  \bitbox[]{1}{\tiny ~ \\ ~ \\ \tt \bf \dots}
  \bitbox[]{3}{}
  \bitbox[]{2}{\tiny ~ \\ ~ \\ 24}
  \bitbox[]{2}{\tiny ~ \\ ~ \\ 23}
  \bitbox[]{2}{\tiny ~ \\ ~ \\ 20}
  \bitbox[]{2}{\tiny ~ \\ ~ \\ 19}
  \bitbox[]{1}{\tiny ~ \\ ~ \\ \tt \bf \dots}
  \bitbox[]{14}{}
  \bitbox[]{1}{\tiny ~ \\ ~ \\ 2}
  \bitbox[]{1}{\tiny ~ \\ ~ \\ 1}
  \bitbox[]{1}{\tiny ~ \\ ~ \\ 0}
  \bitbox[]{2}{\tiny ~ \\ ~ \\ 31}
  \bitbox[]{1}{\tiny ~ \\ ~ \\ \tt \bf \dots}
  \bitbox[]{26}{}
  \bitbox[]{1}{\tiny ~ \\ ~ \\ 2}
  \bitbox[]{1}{\tiny ~ \\ ~ \\ 1}
  \bitbox[]{1}{\tiny ~ \\ ~ \\ 0}
  \\
  \bitbox{8}{\tiny \bus Reply Address}
  \colorbitbox{lightgray}{8}{\scriptsize RSVD}
  \bitbox{16}{Length-1}
  \bitbox{30}{Read Start Address}
  \colorbitbox{lightgray}{1}{\footnotesize \tt Z}
  \colorbitbox{lightgray}{1}{\footnotesize \tt Z}
  \colorbitbox{lightred}{30}{DMA Start Address on Destination Node}
  \colorbitbox{lightgray}{1}{\footnotesize \tt Z}
  \colorbitbox{lightgray}{1}{\footnotesize \tt Z}
  \\
  \bitbox[t]{96}{$\underbrace{\hspace{45em}}_{\text{\small \bus Data}}$}
\end{bytefield}

The first word indicates the \bus address to reply to and the length of the
requested read in words less one. A length of~0 will reply with 1~word of
data.
The second word is the address in memory to read from. The bottom two
bits of the address field are reserved and {\bf must} be transmitted as 0.

The third word is {\bf optional}. If the third word is present, it is
prepended to the reply (generating a message with a
\nameref{cmd:mem-bulk-write} formatted payload). If the third word is omitted,
data is immediately placed on the bus (generating a message with a
\nameref{cmd:mem-stream-write} formatted payload).

\newcommand{\recordMemRead}{%
\begin{mdframed}[backgroundcolor=lightercyan]
  \begin{minipage}[b]{.35\linewidth}
    \mpqrecord command information. Generate the response and send it. When
    the response completes, trigger an interrupt.  If an error occurs part way
    through, the Reg \#250 should indicate the number of words actually
    written. If the optional third word is omitted, Reg \#251--252 are
    undefined.
    \hl{TODO: Use reserved bit(s) to indicate error?}
  \end{minipage}
  \hfill
  \begin{minipage}[b]{.6\linewidth}
    {\bf Reg \#240:}
    \begin{bytefield}{24}
      \bitheader{0-23} \\
      \bitbox{4}{\scriptsize 4~MSB of\\Prefix}
      \bitbox{4}{0011}
      \bitbox{14}{\scriptsize Start Address Read From [15:2]}
      \colorbitbox{lightgray}{1}{\begin{sideways}{\tiny RSV}\end{sideways}}
      \colorbitbox{lightgray}{1}{\begin{sideways}{\tiny RSV}\end{sideways}}
    \end{bytefield}

    \medskip
    {\bf Reg \#241:}
    \begin{bytefield}{24}
      \bitheader{0-23} \\
      \colorbitbox{lightgray}{8}{Reserved}
      \bitbox{16}{\scriptsize Start Address Read From [31:16]}
    \end{bytefield}

    \medskip
    {\bf Reg \#242:}
    \begin{bytefield}{24}
      \bitheader{0-23} \\
      \colorbitbox{lightgray}{8}{Reserved}
      \bitbox{14}{\scriptsize DMA Address Written To [15:2]}
      \colorbitbox{lightgray}{1}{\begin{sideways}{\tiny RSV}\end{sideways}}
      \colorbitbox{lightgray}{1}{\begin{sideways}{\tiny RSV}\end{sideways}}
    \end{bytefield}

    \medskip
    {\bf Reg \#243:}
    \begin{bytefield}{24}
      \bitheader{0-23} \\
      \colorbitbox{lightgray}{8}{Reserved}
      \bitbox{16}{\scriptsize DMA Address Written To [31:16]}
    \end{bytefield}

    \medskip
    {\bf Reg \#244:}
    \begin{bytefield}{24}
      \bitheader{0-23} \\
      \colorbitbox{lightgray}{8}{Reserved}
      \bitbox{16}{\scriptsize Length Read-1}
    \end{bytefield}
  \end{minipage}
\end{mdframed}
}
\recordMemRead

\hl{TODO: Message length isn't passed here, how to know if target address is
included?}

\paragraph{Overflow:} If the starting address field and subsequent length
exceed the memory space, that is a request for address {\tt 0x100000000} would
have been made during the response, the layer controller wraps and continues
sending from address {\tt 0x00000000}.
{\em Tests: \ref{test:mem-bulk-overflow}.}

\paragraph{Interjection Semantics:} If the reply is interjected, the
transaction is aborted and is {\bf not} retried.

\newpage

\subsubsection{Memory Stream Write}
\label{cmd:mem-stream-write}

\begin{bytefield}{9}
  \bitheader{0-7} \\
  \bitbox[rbt]{5}{{\dots}Prefix}
  \bitbox{2}{01}
  \colorbitbox{lightcyan}{2}{XX} \\
  \bitbox[t]{9}{\hspace{-.5em}$\underbrace{\hspace{9.5em}}_{\text{\small \bus Address}}$}
\end{bytefield}
~
\begin{bytefield}[bitwidth=.5em]{40}
%  \bitheader{0,31} \\
  \bitbox[]{2}{\tiny ~ \\ ~ \\ 31}
  \bitbox[]{1}{\tiny ~ \\ ~ \\ \tt \bf \dots}
  \bitbox[]{28}{}
  \bitbox[]{1}{\tiny ~ \\ ~ \\ 0}
  \bitbox[]{2}{\tiny ~ \\ ~ \\ 31}
  \bitbox[]{1}{\tiny ~ \\ ~ \\ \tt \bf \dots}
  \\
  \bitbox{32}{Data}
  \bitbox[ltb]{8}{Data\ldots}
  \\
  \bitbox[t]{40}{$\underbrace{\hspace{24em}}_{\text{\small \bus Data}}$}
\end{bytefield}

\proto nodes have up to four streaming memory channels. Each channel is
controlled by configuration registers (\nameref{cmd:conf-memory-stream}). The
destination of a memory stream write is specified by a combination of channel
selection---the last two bits of the FU\_ID---and the pre-arranged
configuration.

The message payload is all data.  The destination address is automatically
incremented every time a word is written.  There is no limit on the length of
this message.

\newcommand{\recordMemStreamWrite}{%
\begin{mdframed}[backgroundcolor=lightercyan]
  \begin{minipage}[b]{.35\linewidth}
    \mpqrecord command information. When the write completes, generate an
    interrupt.  If an error occurs part way through, the Reg \#250 should
    indicate the number of words actually written.
    \hl{TODO: Use reserved bit(s) to indicate error?}
  \end{minipage}
  \hfill
  \begin{minipage}[b]{.6\linewidth}
    {\bf Reg \#240:}
    \begin{bytefield}{24}
      \bitheader{0-23} \\
      \bitbox{4}{\scriptsize 4~MSB of\\Prefix}
      \bitbox{2}{01}
      \colorbitbox{lightcyan}{2}{XX}
      \bitbox{14}{\scriptsize Start Address Written To [15:2]}
      \colorbitbox{lightgray}{1}{\begin{sideways}{\tiny RSV}\end{sideways}}
      \colorbitbox{lightgray}{1}{\begin{sideways}{\tiny RSV}\end{sideways}}
    \end{bytefield}

    \medskip
    {\bf Reg \#241:}
    \begin{bytefield}{24}
      \bitheader{0-23} \\
      \colorbitbox{lightgray}{8}{Reserved}
      \bitbox{16}{\scriptsize Start Address Written To [31:16]}
    \end{bytefield}

    \medskip
    \textbf{Reg \#242, \#243 not written.}

    \medskip
    {\bf Reg \#244:}
    \begin{bytefield}{24}
      \bitheader{0-23} \\
      \colorbitbox{lightgray}{8}{Reserved}
      \bitbox{16}{\scriptsize Length Written-1}
    \end{bytefield}
  \end{minipage}
\end{mdframed}
}
\recordMemStreamWrite

\paragraph{Unaligned Access:} The behavior of a message ending on a
non-word boundary is undefined.

\paragraph{Overflow:} If more data is received after writing the last
address in memory ({\tt 0xfffffffc}), the destination address wraps and data
continues to be written, beginning at address {\tt 0x00000000}.
{\em Tests: \ref{test:mem-stream-overflow}.}

\paragraph{Interjection Semantics:} Each word of data is written to memory
immediately when it is received. A four-word message would have a
$(1+4)\times32=160$~bit data payload. If 95~bits are received prior to
interjection, only the first word was written to memory. If 96~bits are
received, the first two words were written to memory.
{\em Tests: \ref{test:inj-mem-stream-long}, \ref{test:inj-mem-stream-end}.}

\newpage

\subsection{Broadcast Snooping}
\label{cmd:snoop}
\nameref{sec:channel-0} and \nameref{sec:channel-1} are built into the \bus
protocol which runs below \proto. However, it is possible that a \proto node
may wish to snoop broadcast traffic (in particular,
\nameref{cmd:query-response}).

If broadcast snooping is active \hl{TODO: Configuration register for this.
Possibly some other filters such as channels?}, whenever a broadcast message
is received:

\newcommand{\recordSnoop}{%
\begin{mdframed}[backgroundcolor=lightercyan]
  \begin{minipage}[b]{.35\linewidth}
    \mpqrecord command information.
    Interrupt after message completes.
  \end{minipage}
  \hfill
  \begin{minipage}[b]{.6\linewidth}
    {\bf Reg \#240:}
    \begin{bytefield}{24}
      \bitheader{0-23} \\
      \bitbox{4}{0000}
      \bitbox{4}{\scriptsize Broadcast Channel}
      \bitbox{16}{\scriptsize Broadcast Message [15:0]}
    \end{bytefield}

    \medskip
    {\bf Reg \#241:}
    \begin{bytefield}{24}
      \bitheader{0-23} \\
      \colorbitbox{lightgray}{8}{Reserved}
      \bitbox{16}{\scriptsize Broadcast Message [31:16]}
    \end{bytefield}
  \end{minipage}
\end{mdframed}
}
\recordSnoop

\hl{TODO: Splitting words across two 24-bit registers all of the time is
  getting a little annoying. I'm not really interested in re-visitng the
  24-bit register decision; there are a number of reasons that was a good
  choice. I am curious, however, if we can do something intelligent that maps
  MMIO accesses from the CPU intelligently so that every MMIO read isn't
wasting 8 bits reading nothing.}

\subsection{Undefined Commands}
\label{cmd:undef}
If command with an unknown FU\_ID is received, \proto behavior is completely
undefined.

\medskip
\noindent
\emph{M3 Implementation Note:} Current behavior silently drops messages with
unknown FU\_IDs.

\newpage
\subsection{Summary of Status Registers}
\label{cmd:status-register-summary}

\recordRegWrite

\recordRegRead

\recordMemBulkWrite

\recordMemRead

\recordMemStreamWrite

\recordSnoop

