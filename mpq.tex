\section{\proto: Point-to-Point Message Protocol}
\bus also defines a common point-to-point messaging protocol: \proto.
%For those chips kind enough to mind their P's and Q's on MBus
Nodes are not required to support \proto to be \bus compliant.

\proto defines two classes of data: register data and memory data. \proto
registers have 8~bit addresses and are 24~bits wide. \proto memory has 32~bit
addresses and stores data that is 32~bits wide. The register address space and
memory address space are separate constructs, that is register address {\tt
0x00} need not map to memory address {\tt 0x00000000}.

\subsection{Register Space}
\label{cmd:reg}
\proto register space is an 8~bit addressable array of 24~bit wide registers.
Any undefined bits are treated as RZWI (Read as Zero, Write Ignored).

\subsubsection{Register Write}
\label{cmd:reg-write}

\begin{bytefield}{9}
  \bitheader{0-7} \\
  \bitbox[rbt]{5}{{\dots}Prefix}
  \bitbox{4}{0000} \\
  \bitbox[t]{9}{\hspace{-.5em}$\underbrace{\hspace{9.5em}}_{\text{\small \bus Address}}$}
\end{bytefield}
~
\begin{bytefield}{32}
  \bitheader{0-31} \\
  \bitbox{8}{Address}
  \bitbox{24}{Data}
  \bitbox[lbt]{8}{Address\dots} \\
  \bitbox[t]{40}{$\underbrace{\hspace{42em}}_{\text{\small \bus Data}}$}
\end{bytefield}

Bits~0-23 of the \bus data field are written to the register addressed by
bits~24-31. The write occurs immediately, as soon as the layer controller
receives the message. Multiple registers may be written in a single \bus
transaction by sending multiple data packets. Each 32~bit chunk of data is
treated as if it were an independent transaction.

\paragraph{Unaligned Access:} The behavior of a message ending on a
non-word boundary is undefined.
%
\paragraph{Interjection Semantics:} Each command is applied immediately when
it is received. A four-command message would have a $4\times32=128$~bit data
payload. If 63~bits are received prior to interjection, only the first command
was applied. If 64~bits are received, the first two commands are applied.
{\em Tests: \ref{test:inj-reg-long}, \ref{test:inj-reg-end}.}

\subsubsection{Register Read}
\label{cmd:register-read}

\begin{bytefield}{9}
  \bitheader{0-7} \\
  \bitbox[rbt]{5}{{\dots}Prefix}
  \bitbox{4}{0001} \\
  \bitbox[t]{9}{\hspace{-.5em}$\underbrace{\hspace{9.5em}}_{\text{\small \bus Address}}$}
\end{bytefield}
~
\begin{bytefield}{32}
  \bitheader{0-31} \\
  \bitbox{8}{Start Address\\to Read From}
  \bitbox{8}{Length-1}
  \bitbox{8}{\bus Address\\to Reply To}
  \bitbox{8}{Address to Write\\on Destination} \\
  \bitbox[t]{32}{$\underbrace{\hspace{33em}}_{\text{\small \bus Data}}$}
\end{bytefield}

Bits~24-31 specify the address of the register to be read. Bits~16-23 may be
used to request that multiple registers are sent. This field is a count of the
number of values to be sent less one, that is a value of~0 requests 1~register
is read and a value of 255 requests that all 256 registers are sent. Bits~8-15
are the \bus address the reply is sent to. Bits~0-7 specify the first address
field of the \nameref{cmd:reg-write} response.

The response message is sent to the \bus address specified in bits~8-15 of the
request and its data field is formatted exactly as the
\nameref{cmd:reg-write} command: 8~bit address + 24~bit data. For reads
of more than one register, the address field in the response is incremented
by~1 for each register.

The response will always respond with the requested number of registers. If a
request for register \#256 would have been made, the request wraps and begins
from register \#0. The destination register address wraps similarly.

\paragraph{Note:} The \bus address to reply to {\bf must} be copied
exactly. The FU\_ID is not required to be \nameref{cmd:reg-write}. For
example, if the FU\_ID is \nameref{cmd-mem-stream-write}, the effect is
dumping the current register state to memory on the target addresss.
{\em Tests: \ref{test:reg-dump-restore}.}

\paragraph{Interjection Semantics:} If an interjection occurs during reply,
the transaction is aborted and is {\bf not} retried.

\subsection{Memory Space}
\label{cmd:mem}
\proto memory space is a 32~bit addressable array of 32~bit words of memory.
Any undefined accesses are treated as RZWI (Read as Zero, Write Ignored).
\proto provides two types of memory commands: bulk and stream. A bulk memory
transaction is a wholly self-contained event designed for DMA of large blocks
of memory. Memory streams pre-configure the stream destination address and
omit it from subsequent transactions, relying on the receiver to maintain and
increment a destination pointer. Streams are useful for applications such as
continuous sampling, where multiple, short messages are generated.

\subsubsection{Memory Bulk Write}
\label{cmd:mem-bulk-write}

\begin{bytefield}{9}
  \bitheader{0-7} \\
  \bitbox[rbt]{5}{{\dots}Prefix}
  \bitbox{4}{0010} \\
  \bitbox[t]{9}{\hspace{-.5em}$\underbrace{\hspace{9.5em}}_{\text{\small \bus Address}}$}
\end{bytefield}
~
\begin{bytefield}[bitwidth=.4em]{72}
%  \bitheader{0,31} \\
  \bitbox[]{2}{\tiny ~ \\ ~ \\ 31}
  \bitbox[]{1}{\tiny ~ \\ ~ \\ \tt \bf \dots}
  \bitbox[]{26}{}
  \bitbox[]{1}{\tiny ~ \\ ~ \\ 2}
  \bitbox[]{1}{\tiny ~ \\ ~ \\ 1}
  \bitbox[]{1}{\tiny ~ \\ ~ \\ 0}
  \bitbox[]{2}{\tiny ~ \\ ~ \\ 31}
  \bitbox[]{1}{\tiny ~ \\ ~ \\ \tt \bf \dots}
  \bitbox[]{28}{}
  \bitbox[]{1}{\tiny ~ \\ ~ \\ 0}
  \bitbox[]{2}{\tiny ~ \\ ~ \\ 31}
  \bitbox[]{1}{\tiny ~ \\ ~ \\ \tt \bf \dots}
  \\
  \bitbox{30}{DMA Start Address}
  \colorbitbox{lightgray}{1}{\footnotesize \tt Z}
  \colorbitbox{lightgray}{1}{\footnotesize \tt Z}
  \bitbox{32}{Data}
  \bitbox[ltb]{8}{Data\ldots}
  \\
  \bitbox[t]{72}{$\underbrace{\hspace{33em}}_{\text{\small \bus Data}}$}
\end{bytefield}

The first word received is the address in memory to begin writing to. The
bottom two bits of the address field are reserved and {\bf must} be
transmitted as 0. Subsequent words are treated as data. The first word of data
is written to the specified address. The next word of data is written to
address+4 (the next word in memory) and so on. There is no limit on the length
of this message.

\paragraph{Unaligned Access:} The behavior of a message ending on a
non-word boundary is undefined.

\paragraph{Overflow:} If more data is received after writing the last
address in memory ({\tt 0xfffffffc}), the destination address wraps and data
continues to be written, beginning at address {\tt 0x00000000}.
{\em Tests: \ref{test:mem-bulk-overflow}.}

\paragraph{Interjection Semantics:} Each word of data is written to memory
immediately when it is received. A four-word message would have a
$(1+4)\times32=160$~bit data payload. If 95~bits are received prior to
interjection, only the first word was written to memory. If 96~bits are
received, the first two words were written to memory.
{\em Tests: \ref{test:inj-mem-bulk-long}, \ref{test:inj-mem-bulk-end}.}

\subsubsection{Memory Bulk Read}
\label{cmd:mem-bulk-read}

\begin{bytefield}{9}
  \bitheader{0-7} \\
  \bitbox[rbt]{5}{{\dots}Prefix}
  \bitbox{4}{0011} \\
  \bitbox[t]{9}{\hspace{-.5em}$\underbrace{\hspace{9.5em}}_{\text{\small \bus Address}}$}
\end{bytefield}
~
\begin{bytefield}[bitwidth=.4em]{96}
%  \bitheader{0,31} \\
  \bitbox[]{2}{\tiny ~ \\ ~ \\ 31}
  \bitbox[]{1}{\tiny ~ \\ ~ \\ \tt \bf \dots}
  \bitbox[]{26}{}
  \bitbox[]{1}{\tiny ~ \\ ~ \\ 2}
  \bitbox[]{1}{\tiny ~ \\ ~ \\ 1}
  \bitbox[]{1}{\tiny ~ \\ ~ \\ 0}
  \bitbox[]{2}{\tiny ~ \\ ~ \\ 31}
  \bitbox[]{1}{\tiny ~ \\ ~ \\ \tt \bf \dots}
  \bitbox[]{3}{}
  \bitbox[]{2}{\tiny ~ \\ ~ \\ 24}
  \bitbox[]{2}{\tiny ~ \\ ~ \\ 23}
  \bitbox[]{2}{\tiny ~ \\ ~ \\ 20}
  \bitbox[]{2}{\tiny ~ \\ ~ \\ 19}
  \bitbox[]{1}{\tiny ~ \\ ~ \\ \tt \bf \dots}
  \bitbox[]{16}{}
  \bitbox[]{1}{\tiny ~ \\ ~ \\ 0}
  \bitbox[]{2}{\tiny ~ \\ ~ \\ 31}
  \bitbox[]{1}{\tiny ~ \\ ~ \\ \tt \bf \dots}
  \bitbox[]{26}{}
  \bitbox[]{1}{\tiny ~ \\ ~ \\ 2}
  \bitbox[]{1}{\tiny ~ \\ ~ \\ 1}
  \bitbox[]{1}{\tiny ~ \\ ~ \\ 0}
  \\
  \bitbox{8}{\tiny \bus Reply Address}
  \colorbitbox{lightgray}{4}{\scriptsize RS\\VD}
  \bitbox{20}{Length-1}
  \bitbox{30}{Read Start Address}
  \colorbitbox{lightgray}{1}{\footnotesize \tt Z}
  \colorbitbox{lightgray}{1}{\footnotesize \tt Z}
  \bitbox{30}{DMA Start Address on Destination Node}
  \colorbitbox{lightgray}{1}{\footnotesize \tt Z}
  \colorbitbox{lightgray}{1}{\footnotesize \tt Z}
  \\
  \bitbox[t]{96}{$\underbrace{\hspace{44em}}_{\text{\small \bus Data}}$}
\end{bytefield}

The first word indicates the \bus address to reply to and the length of the
requested read {\em in words} less one. A length of~0 will reply with 1~word
of data.
The second word is the address in memory to read from. The bottom two
bits of the address field are reserved and {\bf must} be transmitted as 0.
The third word specifies the memory address to issue a bulk write to.  This
word is copied verbatim as the first word of the response data, including the
bottom reserved bits.

The response message is formatted exactly as the \nameref{cmd:mem-bulk-write}
command, that is a 32~bit address field followed by a series of sequential
32~bit data fields. In this way, layer~A could request a DMA read from layer~B
that results in a DMA write to layer~C.

\paragraph{Overflow:} If the starting address field and subsequent length
exceed the memory space, that is a request for address {\tt 0x100000000} would
have been made during the response, the layer controller wraps and continues
sending from address {\tt 0x00000000}.
{\em Tests: \ref{test:mem-bulk-overflow}.}

\paragraph{Interjection Semantics:} If an interjection occurs during reply,
the transaction is aborted and is {\bf not} retried.


\subsubsection{Memory Stream Read, Configure, or Alert}
\label{cmd:mem-stream-multi}

\begin{bytefield}{9}
  \bitheader{0-7} \\
  \bitbox[rbt]{5}{{\dots}Prefix}
  \bitbox{4}{0100} \\
  \bitbox[t]{9}{\hspace{-.5em}$\underbrace{\hspace{9.5em}}_{\text{\small \bus Address}}$}
\end{bytefield}
~
\begin{bytefield}[bitwidth=.5em]{64}
%  \bitheader{0,31} \\
  \bitbox[]{1}{\tiny ~ \\ ~ \\ 31}
  \bitbox[]{5}{}
  \bitbox[]{2}{\tiny ~ \\ ~ \\ 24}
  \bitbox[]{1}{\tiny ~ \\ ~ \\ \begin{sideways}{\tiny 23~}\end{sideways}}
  \bitbox[]{1}{\tiny ~ \\ ~ \\ \begin{sideways}{\tiny 22~}\end{sideways}}
  \bitbox[]{1}{\tiny ~ \\ ~ \\ \begin{sideways}{\tiny 21~}\end{sideways}}
  \bitbox[]{1}{\tiny ~ \\ ~ \\ \begin{sideways}{\tiny 20~}\end{sideways}}
  \bitbox[]{2}{\tiny ~ \\ ~ \\ 19}
  \bitbox[]{1}{\tiny ~ \\ ~ \\ \tt \bf \dots}
  \bitbox[]{16}{}
  \bitbox[]{1}{\tiny ~ \\ ~ \\ 0}
  \bitbox[]{2}{\tiny ~ \\ ~ \\ 31}
  \bitbox[]{1}{\tiny ~ \\ ~ \\ \tt \bf \dots}
  \bitbox[]{26}{}
  \bitbox[]{1}{\tiny ~ \\ ~ \\ 2}
  \bitbox[]{1}{\tiny ~ \\ ~ \\ 1}
  \bitbox[]{1}{\tiny ~ \\ ~ \\ 0}
  \\
  \bitbox{8}{\tiny \bus Reply Address}
  \bitbox{1}{\footnotesize 0}
  \colorbitbox{lightgray}{1}{\footnotesize \tt Z}
  \colorbitbox{lightgray}{1}{\footnotesize \tt Z}
  \colorbitbox{lightgray}{1}{\footnotesize \tt Z}
  \bitbox{20}{Length-1}
  \bitbox{30}{Read Start Address}
  \colorbitbox{lightgray}{1}{\footnotesize \tt Z}
  \colorbitbox{lightgray}{1}{\footnotesize \tt Z}
  \bitbox[]{10}{\raggedright ~$\leftarrow$ Read}
  \\
  \bitbox{8}{\tiny Alert \bus Address}
  \bitbox{1}{\footnotesize 1}
  \bitbox{2}{\tiny 01 10 11}
  \bitbox{1}{\begin{sideways}{\tiny DBLB}\end{sideways}}
  \bitbox{20}{Buffer Length-1}
  \bitbox{30}{Write Start Address}
  \colorbitbox{lightgray}{1}{\footnotesize \tt Z}
  \colorbitbox{lightgray}{1}{\footnotesize \tt Z}
  \bitbox[]{12}{\raggedright ~$\leftarrow$ Configure}
  \\
  \bitbox{8}{\tiny Source \bus Address}
  \bitbox{1}{\footnotesize 1}
  \bitbox{2}{\footnotesize 00}
  \colorbitbox{lightgray}{1}{\footnotesize \tt Z}
  \bitbox{20}{Valid Data Length-1}
  \bitbox{30}{Valid Data Start Address}
  \colorbitbox{lightgray}{1}{\footnotesize \tt Z}
  \colorbitbox{lightgray}{1}{\footnotesize \tt Z}
  \bitbox[]{10}{\raggedright ~$\leftarrow$ Alert}
  \\
  \bitbox[t]{64}{$\underbrace{\hspace{30em}}_{\text{\small \bus Data}}$}
\end{bytefield}

Three commands share the {\tt 0100} FU\_ID. Bits~21--23 in the first word
determine whether the command is a read, configure, or alert. A stream read
generates a stream write. A stream configure is an isolated command with no
response. A stream alert is generated in response to a stream write.

\paragraph{Read (Bit 23 == 0):}
\label{cmd:mem-stream-multi-read}
The first word indicates the \bus address to reply to and the length of the
requested read in words less one.
The second word received is the address in memory to read from. The bottom two
bits of the address field are reserved and {\bf must} be transmitted as 0.

The response is sent immediately and the message is formatted exactly as the
\nameref{cmd:mem-stream-write} command, that is a series of sequential 32~bit
data fields.

\subparagraph{Overflow:} If the starting address field and subsequent length
exceed the memory space, that is a request for address {\tt 0x100000000} would
have been made during the response, the layer controller wraps and continues
sending from address {\tt 0x00000000}.
{\em Tests: \ref{test:mem-stream-overflow}.}

\subparagraph{Interjection Semantics:} If an interjection occurs during reply,
the transaction is aborted and is {\bf not} retried.

\paragraph{Configure (Bit 23 == 1, Bits 22-21 in (01,10,11):}
\label{cmd:mem-stream-multi-conf}
Bits~22--21 specify which streaming channel is currently being configured.
The first byte holds the \bus address that an Alert message will be sent to
when this streaming channel's buffer is full.
Bits~19--0 specify the maximum length of the buffer.
The second word specifies the address in memory to begin writing to.
Bit~20 {\tt DBLB} specifies whether double-buffering mode is active. Without
double-buffering, an Alert is generated once the buffer is full and the buffer
cannot be written into again until another Configure is received. With
double-buffering, an Alert is generated once halfway through the buffer and
again at the end of the buffer. At the end of the buffer, the address resets
to the beginning of the buffer and continues to accept writes.

\paragraph{Alert (Bit 23 == 1, Bits 22-21 == 00):}
\label{cmd:mem-stream-multi-alert}
The first byte holds the address of the node generating the Alert.
The second word holds the start of the buffer in the local address space and
the end of the first word specifies the length of the buffer that is valid.


\subsubsection{Memory Stream Write}
\label{cmd:mem-stream-write}

\begin{bytefield}{9}
  \bitheader{0-7} \\
  \bitbox[rbt]{5}{{\dots}Prefix}
  \bitbox{4}{\tiny 0101\\0110\\0111} \\
  \bitbox[t]{9}{\hspace{-.5em}$\underbrace{\hspace{9.5em}}_{\text{\small \bus Address}}$}
\end{bytefield}
~
\begin{bytefield}[bitwidth=.5em]{40}
%  \bitheader{0,31} \\
  \bitbox[]{2}{\tiny ~ \\ ~ \\ 31}
  \bitbox[]{1}{\tiny ~ \\ ~ \\ \tt \bf \dots}
  \bitbox[]{28}{}
  \bitbox[]{1}{\tiny ~ \\ ~ \\ 0}
  \bitbox[]{2}{\tiny ~ \\ ~ \\ 31}
  \bitbox[]{1}{\tiny ~ \\ ~ \\ \tt \bf \dots}
  \\
  \bitbox{32}{Data}
  \bitbox[ltb]{8}{Data\ldots}
  \\
  \bitbox[t]{40}{$\underbrace{\hspace{19em}}_{\text{\small \bus Data}}$}
\end{bytefield}

The message payload is all data. The address that the data is written to must
have been previously configured by \nameref{cmd:mem-stream-multi-conf}. The
destination address is automatically incremented every time a word is written.
There is no limit on the length of this message.

\paragraph{Unaligned Access:} The behavior of a message ending on a
non-word boundary is undefined.

\paragraph{Overflow:} If more data is received after writing the last
address in memory ({\tt 0xfffffffc}), the destination address wraps and data
continues to be written, beginning at address {\tt 0x00000000}.
{\em Tests: \ref{test:mem-stream-overflow}.}

\paragraph{Interjection Semantics:} Each word of data is written to memory
immediately when it is received. A four-word message would have a
$(1+4)\times32=160$~bit data payload. If 95~bits are received prior to
interjection, only the first word was written to memory. If 96~bits are
received, the first two words were written to memory.
{\em Tests: \ref{test:inj-mem-stream-long}, \ref{test:inj-mem-stream-end}.}
