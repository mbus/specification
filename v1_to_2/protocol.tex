\svnInfo $Id$

\section{Introduction}

This document is an exploration of an idea, it assumes you are familiar with
\bus v1.0.

\section{Changes}
Loosely ordered as a series of changes from v1.0.

\subsection{Double Data Rate}
We dismissed this when we concluded that latching on both the positive and
negative edges was not feasible in a general sense. What we neglected,
however, was so long as we could design a system where each flop latched only
on the positive edge {\bf or} the negative edge (never both), then the option
is back on the table (since what is latching on negative clock edges but an
inverter and a clock that is 180$^\circ$ out of phase?).

The insight is that there is no need for a {\sc drive1} or {\sc drive2} {\em
state}. The only transition that occurs on {\sc drive*} edges is a new bit
being placed onto the bus. Logic and decisions are all consequences of the
{\sc latch*} states. If you control and clock the MUX that drives {\tt DOUT}
with the 0$^\circ$ clock, but feed the data entry to the MUX with a FIFO
clocked by a 180$^\circ$ clock, the effect is the value on the data line
changing on the negative edge of clock.

This change ignores details of arbitration, termination, acknowledgment, and
reset for now, it is an optimization meant for during data transmission.

\subsection{Removing the double-latching}
The motivation for a separate {\sc latch1} and {\sc latch2} was to (in a
purely digital manner) allow for arbitrary reset events (interruption /
preemption) and to very cleanly distinguish Reset, Stop, and Acknowledge
sequences from the data stream. But there is another way\ldots

In this version, both the clock and the data are connected in a buffered loop
(yes this does cause some issues for injection, TBD). When any node wishes to
interrupt a transmission---where interrupt is defined as both ending a
transmission (stop bit) or forcing an abort---it stops forwarding the {\tt
CLK} signal. The control layer will detect that its {\tt CLK} input has not
gone low at the next {\tt CLK} output. In response, the control layer will
start toggling the {\em data} lines. Each member node will have a small
counter that is clocked from {\em data} and asynchronously reset by {\tt CLK}.
During normal transmission, the counter should never be able to count higher
than 1 as the clock is constantly resetting it. If the counter reaches 2 then
the clock and data loops {\em switch roles}.

\begin{figure}[ht]
\noindent\makebox[\textwidth]{%
  \footnotesize
  \begin{tikztimingtable}[timing/wscale=3.0,timing/slope=.3]
    %          10011 EE EE IIII SS X00110 I
    CLK In   & CCCCC CC CC CHHH HC CCCCCC C\\
    CLK Out  & CCCCC CC CC CHHH HC CCCCCC C\\
    Data In  & HHLLH HH HH HCCC CH HLLHHL H\\
    Data Out & HHLLH HH HH HCCC CH HLLHHL H\\
             & {1D{Data 1}}{2D{Data 0}}{2D{Data 1}}
               {2D{Data 1}}{2D{Data 1}}{4D{Interrupt}}{2D{Switch Role}}
               {2D{Control 0}}{2D{Control 1}}{2D{Control 0}}{1D{Idle}} \\
    \\
    % TX Node
    CLK In   & CCCCC CC CC CHHH HC CCCCCC C\\
    CLK Out  & CCCCC CC LL LLLL LL CCCCCC C\\
    Data In  & HHLLH HH HH HCCC CH HLLHHL H\\
    Data Out & HHLLH HH HH HCCC CH HLLHHL H\\
             & {1D{Data 1}}{2D{Data 0}}{2D{Data 1}}
               {4D{Req Interrupt}}{4D{Interrupt}}{2D{Switch Role}}
               {2D{Control 0}}{2D{Control 1}}{2D{Control 0}}{1D{Idle}} \\
    \\
    CLK In   & CCCCC CC LL LLLL LL CCCCCC C\\
    CLK Out  & CCCCC CC LL LLLL LL CCCCCC C\\
    Data In  & HHLLH HH HH HCCC CH HLLHHL H\\
    Data Out & HHLLH HH HH HCCC CH HLLHHL H\\
             & {1D{Data 1}}{2D{Data 0}}{2D{Data 1}}
               {8D{Interrupt}}{2D{Switch Role}}
               {2D{Control 0}}{2D{Control 1}}{2D{Control 0}}{1D{Idle}} \\
    \\
    % CTL Node
    CLK In   & CCCCC CC LLL LLL LL CCCCCC C\\
    CLK Out  & CCCCC CC CCC HHH HC CCCCCC C\\
    Data In  & HHLLH HH HHH CCC CH HLLHHL H\\
    Data Out & HHLLH HH HHH CCC CH HLLHHL H\\
    Int Clk  & CCCCC CC CCC CCC CC CCCCCC C\\
             & {1D{Data 1}}{2D{Data 0}}{2D{Data 1}}
               {4D{Enter Interrupt}}{4D{Interrupt}}{2D{Switch Role}}
               {2D{Control 0}}{2D{Control 1}}{2D{Control 0}}{1D{Idle}} \\
  \extracode
    \begin{pgfonlayer}{background}
      \begin{scope}[semithick,dashed]
        \vertlines[color=red]{3,9,...,27}
        \vertlines[color=gray]{6,12,...,15}
        \vertlines[color=blue]{39,45,...,\twidth}
        \vertlines[color=OliveGreen]{42,48,...,\twidth}

        \filldraw[yellow,opacity=.25] (15, 2) rectangle (27, -45);
      \end{scope}
    \end{pgfonlayer}

    \draw[red,thick]    (21,  -2.5) ellipse (1.25 and 4.5);
    \draw[red,thick]    (27,  -2.5) ellipse (1.25 and 4.5);
    \draw[blue,thick]   (39,  -2.5) ellipse (1.25 and 4.5);
    \draw[green,thick]  (39, -26.5) ellipse (1.25 and 4.5);
    \draw[cyan,thick]   (24, -36.5) ellipse (1.25 and 2.5);
    \draw[cyan,semithick,dashed] (24,-44.5) to (24,-33.25); % .25 better dash
    \node at (18, -7.5)  {\huge\color{blue} X};
    \node at (24, -7.5)  {\huge\color{blue} X};
    \node at (18, -31.5) {\huge\color{green} $\checkmark$};
    \node at (24, -31.5) {\huge\color{green} $\checkmark$};

    \begin{scope}
      [font=\bf\sffamily,shift={(-5.5em,-1.5)},anchor=east,color=blue]
      \node [rotate=45] at (  0,   0) {1};
      \node [rotate=45] at (  0, -12) {2 (TX)};
      \node [rotate=45] at (  0, -24) {3};
      \node [rotate=45] at (  0, -36) {Ctl};
    \end{scope}

    \foreach \n [evaluate=\n as \l using int((\n-1)/3)] in {3,6,...,\twidth}
      \draw (\n,-46.5) -- +(0,-.2)
        node [below,inner sep=2pt] {\scalebox{.75}{\footnotesize\l}};
  \end{tikztimingtable}
}
  \caption{
    Detail of the Interrupt entry procedure. In this example, Node~2 is the
    TX node, thus when it elects to enter Interrupt it is already forwarding
    its data lines. Node~2 decides to enter Interrupt
    at time~4. At time~6 Node~2 suppresses the clock edge, requesting
    Interrupt. The control layer detects this at time~7\protect\footnotemark.
    At this point the control node begins toggling the {\em data} lines to act
    as a clock. After two data pulses, all nodes will enter Interrupt. A few
    control bits are sent, after which the bus returns to Idle.
  }
  \label{fig:interrupt}
\end{figure}
\footnotetext{This detection must occur on the falling edge since the control
layer requires an egde to sample on and it is sampling its {\tt CLK\_IN} pin
detecting when it is not high. A simple detector can sample this every edge
asserting a signal which is sampled by the controller's FSM at time 8,
transitioning the controller into interrupt mode.}

Figure~\ref{fig:interrupt} demonstrates an entry into interrupt. Note the
careful subtelty depending where a node is physically located. At time~6
Node~1 will record an extra data bit that Node~3 will not record. This is
resolved at time~10 when Interrupt is entered. When entering Interrupt, a node
must sample its {\tt CLK} line. If {\tt CLK} is high, the node should discard
the most recently recorded bit. The control node must wait until the third
data edge (time~12) to begin forwarding the {\tt CLK} line to ensure a stable
value is sampled for the possible bit-deletion. The control node should rely
on the incoming {\tt CLK} to latch data bits, otherwise it would also have to
discard a bit recorded at time~6.

Arbitration of interrupt entry acts as other arbitration has before.
Considering Figure~\ref{fig:interrupt}, had Node~3 decided at the same time as
Node~2 (time~4) that it wished to interrupt it would never see another {\tt
CLK} edge before the data pulses arrived, causing the node to lose
arbitration.

